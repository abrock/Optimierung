\chaptr{10 Indirekter Ansatz der optimalen Steuerung}

\msection{Optimalsteuerungsproblem}

\begin{align*}
\min &\int\limits_{t_0}^\tend L(t,y(t),u(t)) \dd t + \Phi(\tend, y(tend)) \quad (10.1) \\
\text{s.\,t. } \dot y(t) &= f(t, y(t), u(t)) \quad (10.2) \\
u & \in \Omega \quad (10.3) \\
y(t_0) &= y_0, r(\tend, y(\tend)) = 0 \quad (10.4) \\
b(t,y(t), u(t)) & \geq 0 \quad (10.5)
\end{align*}

Variablen: $u,y$, ggf. $\tend, t_0, y_0$.

Grundannahme: $L, \Phi, f, r, b$ sind $C^3$

\msubsection{Definition 10.1}

Eine Steuerung $u$ heißt zulässig, wenn $y$ auf $[t_0, \tend]$ existiert, so dass $(u,y)$ die Nebenbedingungen (10.2) - (10.5) erfüllt. Die Steuerung $u^*$ heißt optimal, wenn $u^*$ zulässig ist und $(u^*, y^*)$ den kleinsten Zielfunktionswert aller zulässigen $(u,y)$ ergibt.

\msubsection{Notewendige Optimalitätsbedingung}

Satz 10.2: Pontryaginsches Maximumsprinzip (1959).

Sei $u^*$ eine optimale Steuerung mit zugehörigem $y^*$. Definiere:

\bitm
\item Hamiltonfunktion: $H(t,y,u,\lambda) := - \gamma L(t,y,u) + \lambda^T f(t,y,u)$ (10.6) \\
\item Erweiterte Zielfunktion $\Psi(\tend, y(\tend), \alpha) = \gamma \Phi(\tend, y(\tend)) + \alpha^T r(\tend, y(\tend))$ (10.7). $\lambda(t) \in \R^{n_y}, \alpha \in \R^{n_r}, \gamma \in \R$, \obda $\gamma = 1$ falls $\gamma \neq 0$.
\eitm

Dann gilt: $\exists \gamma \geq 0, \alpha \in \R^{n_y}, \lambda : [t_0, \tend] \to \R^{n_y}$ und $(\alpha, \lambda(t)) \neq 0$ ($\alpha, \gamma$ heißen "`Lagrange-Parameter"' bzw. adjungierte Variablen), so dass:

\bitm
\item $\lambda(t)$ erfüllen die adjungierten Differentialgleichungen $\dot\lambda(t)^T = \gamma L_y (t, y^*, u^*(t)) - \lambda(t)^T f_y(t, y^*(t), u^*(t)) = - H_y (t, y^*(t), u^*(t), \lambda(t))$ (10.8).
\item Transversalitätsbedingung: $\lambda(\tend)^T = - \Psi_y (\tend, y^*(\tend), \alpha)$ (10.9). Falls $\tend$ frei: $H(\tend, y^*(\tend), u^*(\tend), \lambda(\tend)) = \Psi_\tend(\tend, y^*(\tend), \alpha)$ (10.10)
\item $u^*$ erfüllt das Maximumsprinzip: $H(t, y^*(t), u^*(t), \lambda(t)) \geq H(t,y^*(t), v(t), \lambda(t))$ fast überall auf $[t_0, \tend]$ für $v \in \Omega$ beliebig. D.\,h. $u^*$ ist die Lösung von $\max_{v\in \Omega} H(t, y^*(t), v(t), \lambda(t)) = H(t, y^*(t), u^*(t), \lambda(t))$ fast überall (10.12) bzw. $u^*(t) = \arg\max_{v\in \Omega} H(t, y^*(t), v(t), \lambda(t))$ fast überall (10.13).
\eitm 

\msubsection{Anwendungsbeispiel: Energieoptimaler Raketenwagen}

\begin{align*}
\dot s &= v \quad s(0) = 0 \quad s(T) = d \\
\dot v &= u \quad v(0) = 0 \quad v(T) = 0 \quad u \in [-1,1]\\
\min & \int\limits_0^T \frac 12 u(t)^2 \dd t \\
y :&= \bpm s \\ v \epm \\
y(0) &= y_0 &= \bpm 0 \\ 0 \epm \\
r(T, y(T)) &= \bpm s(T)-d \\ v(T) \epm = \bpm 0 \\ 0 \epm \\
t_0 &= 0 \\
\tend &= T \text{ fest} \\
\end{align*}

Hamiltonfunktion: 

\begin{align*}
H &= - \frac{u^2}{2} + \lambda_s v + \lambda_v u = H(v, \lambda_s, \lambda_v, u)  \\
\intertext{adjungierte Differentialgleichung:} \\
\dot \lambda_s &= - \frac{\partial H}{\partial s} = 0 \RA \lambda_s \text{ konstant} \\
\dot \lambda_v &= -\frac{\partial H}{\partial v} = - \lambda_s \RA \lambda_v \text{ Gerade} \\
\end{align*}

Maximumsprinzip

\begin{align*}
H_u &= - u + \lambda_v = 0 \RA u = \lambda v \\
u \text{ Gerade: } u &= at+b \\
\RA v \text{ Parabel: } v &= \frac a2 t^2 + tb + c, \quad c = 0 \\
\RA s \text{ kubisches Polynom: } s &= \frac a6 t^3 + \frac b2 t^2 + c' \quad c' = 0 \\
\end{align*}

$T$ einsetzen, $a,b$ ausrechnen:

\begin{align*}
a &= - \frac{12 d}{T^3} \\
b &= 6 \frac d{T^2} \\
\end{align*}

TODO: Bildchen malen mit Geraden (fallend), Parabeln (konkav)

Falls $b \leq 1$ ist diese Lösung zulässig $\LRA 6d \leq T^2$. Was passiert, wenn $T < \sqrt{6d}$, d.\,h. $6d > T^2$? Genauer: 

\begin{align*}
\max_{u\in [-1,1]} H(v, \lambda_s, \lambda_v, u) &= - \frac{u^2}2 + \lambda_s v + \lambda_v u \\
\end{align*}

Fälle:

\bitm
\item $u = \lambda v$, falls $\lambda v \in (-1,1)$
\item $u = -1$, falls $\lambda v \leq -1$
\item $u = 0$, falls $\lambda v \geq 1$
\eitm 

$H$ ist hier konkav in $u$, d.\,h. $H_{uu} \leq 0$. Es gilt: Falls $y, \lambda$ stetig sind, $H_{uu}$ strikt negativ definit und $\Omega$ konvex $\RA u^*(t)$ stetig. Annahme: Struktur der Lösung:
Erst $u=1$ von $0$ bsi $t_1$, dann $u = \lambda v$ von $t_1$ bis $t_2$, dann $u = -1$ von $t_2$ bis $T$.

Erfüllt Maximumprinzip falls $b < 1$, falls $b>1$ wegen Konkavität von $H$.

Zurück zum Raketenwagen:

\begin{align*}
v(t) &= \begin{cases} t & t \in [0, t_1] \\ a t^2/2 + bt + c & t \in [t_1, t_2] \\T -t & t \in [t_2, T] \end{cases} \
s(t) &= \int\limits_0^t v(\tau) \dd \tau \\
\end{align*}

Bestimmung der Variablen: $a,b,c,t_1, t_2$. Stetigkeit von $v$: $a t_1^2 /2 + b t_1 + c = t_1$, $a t_2^2/2 + b t_2 + c = T -t_2$. Randbedingung: $s(T) = d$. Stetigkeit von $u$: $a t_1 + b = 1$, $a t_2 +b = -1$. 5 (nichtlineare) Bedingungen für 5 Variablen.

Formulierung als Mehrpunkt-Randwertproblem:

\begin{align*}
u(t) &= \begin{cases} 1 & 0 \leq t \leq t_1 \\ \lambda v & t_1 \leq t \leq t_2 \\ -1 & t_2\leq t \leq T \end{cases}
\dot s &= v \\
\dot \lambda_v &= -\lambda_s \\
\dot v &= u \\
\dot \lambda_s &= 0 \\
\end{align*}

Randbedingungen:

\begin{align*}
s(0) &= 0 \\
s(T) &= d \\
v(0) &= 0 \\
v(T) &= 0 \\
\end{align*}

Innere-Punkt-Bedingungen:

\begin{align*}
\lambda v(t_s) &=  1 \\
\lambda v(t_2) &= -1 \\
\end{align*}

Das ist ein Randwertproblem mit 4 Differentialgleichungen, 2 zu bestimmenden Größen und 4 Randbedingungen, ist also im Prinzip lösbar, numerisch mit einem Randwertproblemlöser.


\msubsection{Prinzip der indirekten Methode}

\bitm
\item Stelle adjungierte Gleichung auf
\item Stelle Transversalitätsbedingungn auf
\item Werte Maximumsprinzip aus: vermute Struktur der Steuerung
\item Formuloere als Mehrpunktrandwertproblem
\item Löse durch numerische "`Diskretisierung"'
\eitm

(First optimize, then discretize).

Beweis von Satz 10.2 (Maximumprinzip):

Wir beweisen das Maximumprinzip zunächst für das Problem

\begin{align*}
\min_{u,y} &\int\limits_{t_0}^\tend L(t,y,u) \dd t + \Phi(\tend, y(\tend))  =: J(y,u) \quad (10.14) \\
\dot y &= f(t,y,u) \\
y(t_0) &= y_0 \\
t & \in [t_0, \tend] \\
u &\in \Omega
\end{align*}

d.\,h. ohne $r$, $b$ und mit $t_0, \tend$ fest. Grundidee: Konstruktion zulässiger Konkurrenten (Variationen) zur optimalen Lösung $u^*$ mit Antwort $y^*$.

$\ov u(t) := u^*(t) + \delta u(t)$ mit zugehöriger Antwort $\ov y(t) := y^*(t) + \delta y(t)$, die wie folgt konstruiert wird:

\bitm
\item Wähle $\delta u \RA \ov u$
\item Löse DGL $\RA \ov y$
\item $\delta y = \ov y - y^*$
\eitm

$\delta y$ löst die DGL $\delta \dot y = \dot{\ov y} - \dot y^* = f(t,y^* + \delta y, u^* + \delta u) - f(t,y^*,u^*)$.

Aus Stabilitätssatz (Satz 1.1) folgt:

\[ \|\delta y\| = \mathcal O(\|\delta u\|) \text{ Norm im Funktionenraum } \]

Konstruktion der Störung: Beispiel 1: $\Delta u(t)$ beliebig, $\delta_\eps u(t) := \eps \Delta u(t)$, $\eps > 0$ klein, so dass

\begin{align*}
\ov u &= u^* + \eps \Delta u \in \Omega \RA \|\delta_\eps u \| = \mathcal O(\eps) \\
\delta_\eps \dot y(t) &= f(t,y^* + \delta_\eps y, u^* + \delta_\eps u) - f(t,y^*, u^*) \\
&= \frac{\partial f}{\partial y} \delta_\eps y(t) + \frac{\partial f}{\partial u} \delta_\eps u(t) + \mathcal O(\delta_\eps^2) \\
\delta_\eps y(t_0) &= 0 \\
A(t) :&= \frac{\partial f}{\partial y} \\
B(t) :&= \frac{\partial f}{\partial u} \\
\delta_\eps y(t) &= \int\limits_{t_0}^t A(\tau) \delta_\eps y(t\tau) + B(\tau) \delta_\eps u(\tau) \dd \tau + \mathcal O(\delta_\eps^2) \\
\RA \|\delta_\eps y\| &= \mathcal O(\eps) \\
\end{align*}

Beispiel:

\begin{align*}
\ov u(t) &= \begin{cases} u^*(t) & \text{ falls } t \notin [\hat t, \hat t + \eps) \\ v \in \Omega \text{ bel. } & t \in [\hat t, \hat t + \eps) \end{cases} \\
\|\delta_\eps u\| &= \|\ov u - u^*\| = \mathcal O(\eps) \\
\end{align*}

Im Sinne einer Integralnorm (nicht $\|\cdot\|_\infty$).

Setze Variation in problem (10.14) ein:

\begin{align*}
0 & \leq J(\ov y, \ov u) - J(y^*, u^*) \\
&= \Phi(\tend, \ov y(\tend)) - \Phi(\tend, y^*(\tend)) + \int\limits_{t_0}^\tend L(t, \ov y, \ov u) - L(t, y^*, u^*) \dd t \\
&+ \int\limits_{t_0}^\tend \lambda(t)^T \underbrace{ (\dot{\ov y} - f(t,\ov y, \ov u) + f(t,y^*, u^*) - \dot y^*) }_{=0 \text{ für bel. } \lambda(t) \text{ und bel. } \ov u \text{ mit zug. } \ov y } \dd t \\
&= \Phi(\tend, \ov y(\tend)) - \Phi(\tend, y^*(\tend)) \\
&+ \int\limits_{t_0}^\tend -L(t,y^*, u^*) + \lambda^T f(t, y^*, u^*) + L(t, \ov y, \ov u) - \lambda^T f(t, \ov y, \ov u) \dd t \\
&+ \int\limits_{t_0}^\tend \lambda(t)^T \delta \dot y (t) \dd t \\
&= \Phi(\tend, \ov y(\tend)) - \Phi(\tend, y^*(\tend)) \\
&+ \int\limits_{t_0}^\tend H(t, y^+, u^*, \lambda) - H(t, \ov y, \ov u, \lambda) \dd t \\
&+ \l[ \lambda(t)^T \delta y(t) \r]_{t_0}^\tend - \int\limits_{t_0}^\tend \dot \lambda(t)^T \delta y(t) \dd t \\
\end{align*}

Taylorentwicklung von $H$, erst nach $y$ um $y^*$, dann nach $u$ um $u^*$.

\begin{align*}
&= \frac{\partial \Phi}{\partial y} (\tend, y^*(\tend)) \delta y(\tend) + \mathcal O(\eps^2) \\
&+ \int\limits_{t_0}^\tend H(t, y^*, u^*, \lambda) - H(t, y^*, \ov u, \lambda) - \frac{\partial H}{\partial y} (t, y^*, \ov u, \lambda) \delta y(t) + \mathcal O(\eps) \dd t \\
& + \lambda(\tend)^T \delta y(\tend) - \lambda(t_0)^T \underbrace{\delta y (t_0)}_{=0} - \int\limits_{t_0}^\tend \lambda(t)^T \delta y(t) \dd t \\
&= \l( \frac{\partial \Phi}{\partial y} (\tend, y^*(\tend)) + \lambda(\tend)^T \r) - \int\limits_{t_0}^\tend \dot \lambda(t)^T \delta y (t) \dd t + \mathcal O(\eps^2) \\
&+ \int\limits_{t_0}^\tend H(t, y^*, u^*, \lambda)- H(t, y^*, \ov u, \lambda) - \frac{\partial H}{\partial y} (t, y^*, u^*, \lambda) \delta y(t) - \underbrace{\frac{\partial^2 H}{\partial u\partial y} (t, y^*, u^*, \lambda) \delta u \delta y}_{= \mathcal O(\eps^2)} + \mathcal O(\eps^2) \\
&= (\lambda(\tend)^T + \frac{\partial \Psi}{\partial y} (\tend, y^*(\tend)) \delta y(\tend)) \\
& + \int\limits_{t_0}^\tend (-\dot \lambda(t)^T - \frac{\partial H}{\partial y} (t, y^*, u^*, \lambda)) \delta y(t) \dd t \\
& + \int\limits_{t_0}^\tend H(t, y^*, u^*, \lambda) - H(t, y^*, \ov u, \lambda) \dd t + \mathcal O(\eps^2) 
\end{align*}

Wähle $\lambda(t)$, das Transversalitätsbedingung und adjungierte Differentialgleichung erfüllt. $\lambda(t)$ ist eindeutig bestimmt.

\begin{align*}
\RA 0 &\leq \int\limits_{t_0}^\tend H(t, y^*, u^*, \lambda) - H(t, y^*, \ov u, \lambda) \dd t + \mathcal O(\eps^2) \\
\end{align*}

Für Beispiel 2:

\begin{align*}
0 & \leq \int\limits_{\hat t}^{\hat t + \eps} H(t, y^*, u^*, \lambda) - H(t, y^*, v, \lambda) \dd t + \mathcal O(\eps^2) \\
\end{align*}

Lasse $\eps$ gegen Null gehen:


\begin{align*}
0 & \leq \eps(H(\hat t, y^*(\hat t), u^*(\hat t), \lambda(\hat t)) - H(\that, y^*(\that), v(\that), \lambda(\that)) + \mathcal O(\eps^2) \\
0 & \leq H(\that, y^*(\that), u^*(\that), \lambda(\that)) - H(\that, y^*(\that), v(\that), \lambda(\that)) + \mathcal O(\eps) \\
\RA H(\that, y^*(\that), u^*(\that), \lambda(\that)) & \geq H(t, y^*(t), v(t), \lambda(t)) \forall v \in \Omega \forall t \in [t_0, \tend] \\
\to & \text{Maximumprinzip}
\end{align*}

\msubsection{Bemerkung 10.3: Variation, Beispiel 1}

\begin{align*}
\delta_\eps u(t) &= \eps \Delta u(t) \\
0 & \leq \int\limits_{t_0}^\tend H(t, y^*, u^*, \lambda) - H(t, y^*, u^* + \delta_\eps u, \lambda) \dd t + \mathcal O(\eps^2) \\
&= \int\limits_{t_0}^\tend \frac{\partial H}{\partial u} (t, y^*, u^*, \lambda) ( - \eps \Delta u(t)) \dd t + \mathcal O(\eps^2)
\end{align*}

Wähle $\Delta u = \eta \cdot \frac{\partial H^T}{\partial u}$, $\eta > 0$ hinreichend klein, so dass $u^* + \delta u \in \Omega$.

\begin{align*}
\cdots &= \int\limits_{t_0}^\tend - \eps \eta \frac{\partial H}{\partial u}{\partial H^T}{\partial u} \dd t + \mathcal O(\eps^2) \\
\eps \to 0 & \RA \int\limits_{t_0}^\tend \frac{\partial H}{\partial u} \frac{\partial H}{\partial u}^T \dd t = 0 \\
\RA 0 &= \frac{\partial H}{\partial u} (t, y^*(t), u^*(t), \lambda(t)) = 0 \text{ fast überall } \\
\end{align*}

\begin{align*}
H(t,y,u,\lambda( :&= -L(t,y,u)+\lambda^T f(t,y,u) \\
\Psi(\tend, y(\tend), \alpha) :&= \Phi(\tend, y(\tend)) + \alpha^T r(\tend, y(\tend)) \\
\end{align*}

Seien $u^*, y^*$ optimal, dann existieren $\lambda, \alpha$ so dass:

\begin{align*}
\dot \lambda(t)^T &= -\frac{\partial H}{\partial y} (t,y^*(t), u^*(t), \lambda(t)) \\
\lambda(\tend) &= - \frac{\partial \Psi}{\partial y} (\tend, y^*(\tend), \alpha) \\
H(\tend, y^*(\tend), u^*(\tend), \lambda(\tend)) &= \frac{\partial \Psi}{\partial t} (\tend, y^*(\tend), \alpha) \\
\end{align*}

$u^*$ erfüllt $H(t,y^*(t), u^*(t), \lambda(t)) \geq H(t, y^*(t), u^*(t), v(t), \lambda(t))$ $\forall v \in \Omega$.

\msubsection{Bemerkung 10.4: Maximumprinzip mit Randbedingungen}

\begin{align*}
0 & \leq J(\ov y, \ov u) - J(y^*, u^*) + \int\limits_\tn^\tend \lambda^T(\dot{\ov y} - f(t,\ov y,\ov u)+ \dot y^* + f(t,y^*, u^*)) \dd t \\
&+ \alpha^T (\underbrace{r(\tend, \ov y(\tend)) - r(\tend, y^*(\tend))}_{=0}) \\
\end{align*}

Dabei sind $u^*, y^*$ optimal und $\ov u = u^* + \delta u$, $\ov y = y^* + \delta y$ sind zulässige Konkurrenten.
Taylorentwicklung:

\begin{align*}
0 & \leq \l( \frac{\partial \Phi}{\partial y} (\tend, y^*(\tend)) + \lambda(\tend)^T \r) \delta y(\tend) \\
&+ \int\limits_\tn^\tend \l( -\dot \lambda(t)^T - \frac{\partial H}{\partial y} (t,y^*,u^*,\lambda)\r) \delta y(t) \dd t \\
& + \int\limits_\tn^\tend H(t,y^*,u^*,\lambda) - H(t, y^*, \ov u, \lambda) \dd t\\
&+ \alpha^T \frac{\partial r}{\partial y} (\tend, y^*(\tend)) \delta y(\tend) + \text{ Terme höherer Ordnung } 
\end{align*}

Wenn wir $\alpha, \lambda$ haben, die folgendes erfüllen:

\bitm
\item adjungierte Differentialgleichung
\item Transversalitätsbedingung: \[ \lambda(\tend)^T = \l( -\frac{\partial \Phi}{\partial y} (\tend, y(\tend)) - \alpha^T \frac{\partial r}{\partial y} (\tend, y(\tend))\r) \]
\eitm

folgt wie oben das Maximumsprinzip.

Nicht gezeigt: Existenz von $\alpha$ und $\lambda$

\bemerkung{10.5: freie Endzeit $\tend$}

Es gibt eine optimale Endzeit $\tend^*$, eine optimale Steuerung $u^*$ und die zugehörige $y^*$.
$\delta \tend := \ov \tend - \tend^*$
zu $\ov u$ und $\ov y$. Setze $\tend^*$ und $\ov \tend$ analog zu (10.15) an, Taylorentwicklung auch in $\tend$ um $\tend^*$. Zusätzliche Terme sind dann:

% eieiei, immer wenn's spannend wird ist die Tafel zu Ende.

\begin{align*}
\frac{\partial \Phi}{\partial t} (\tend^*, y(\tend^*)) + \alpha^T \frac{\partial r}{\partial t} (\tend^*, y(\tend^*)) \delta \tend \\
&+ \underbrace{\int\limits_{\tend^*}^{\tend^*+\delta\tend} L(t,\ov y, \ov u) - \lambda(t)^T f(t, \ov y, \ov u) \dd t}_{= - H(\tend, y(\tend))} \\
&+ \underbrace{\int\limits_{\tend^*}^{\tend^*+\delta\tend}\lambda(t)^T \dot{\ov y}(t) \dd t}_{= 0 \text{ durch Wahl von $\lambda\equiv 0$ auf $[\tend^*, \tend^*+\delta\tend]$}} \\
&+ \text{Terme höherer Ordnung} + TODO
\end{align*}

Wennes $\alpha, \lambda$ gibt, die die adjungierte Differentialgleichung, obige Transversalitätsbedingung und $H(\tend^*, y^*(\tend^*),u^*(\tend^*),\lambda(\tend^*)) = \frac{\partial \Phi}{\partial t} (\tend^*, y^*(\tend^*)) + \alpha^T \frac{\partial r}{\partial t} (\tend^*, y^*(\tend^*))$ erfüllen, folgt das Maximumprinzip.

Andere Möglichkeit: Transformation auf feste Endzeit.

Beispiel:

Zielfunktional: $\min \tend$ ($\tn = 0$)

\bitm
\item a) $\Phi = \tend$, $L=0$
\item b) $\Phi = 0$, $L=1$
\eitm

Randbedingung $r(y(\tend))$ hängt nicht explizit von $\tend$ ab.

Transversalitätsbedingungen:

\bitm
\item a) $H = \lambda^T f(t,y,u)$, $H(\tend)=1$
\item b) $H = -1 + \lambda^T f(t,y,u)$, $H(\tend) = 0$
\eitm

Transversalitätsbedingung zum Eliminieren von $\alpha$:

\begin{align*}
\lambda(\tend) &= -\frac{\partial \Phi}{\partial y} (\tend, y(\tend))^T - \frac{\partial r}{\partial y} (\tend, y(\tend))^T \alpha \\
H(\tend) &= \frac{\partial \Phi}{\partial t} + \frac{\partial r}{\partial t}^T \alpha  \text{ Wenn $\tend$ frei} 
\end{align*}

Wenn $\frac{\partial r}{\partial y}^T$ Vollrang hat, können wir die $\alpha$ eliminieren:

\begin{align*}
\frac{\partial r}{\partial y}^T &= \bpm A_1 \\ A_2 \epm
\end{align*}

mit $A_1$ regulär (ggf. nach Zeilenpermutationen)

\begin{align*}
\alpha &= A_1^{-1} \bpm -\lambda_I & - \frac{\partial \Phi}{\partial y_I} \epm \\
\lambda_I &= \bpm \lambda_1 \\ \hdots \\ \lambda_\nu \epm \\
\lambda_{II} &= \bpm \lambda_{\nu+1} \\ \hdots \\ \lambda_n \epm \\
\lambda_{II} &= - \frac{\partial \Phi}{\partial y_{II}}^T - A_2 \alpha, \text{ $\alpha$ einsetzen} \\
\end{align*}

Es fallen die $\alpha$ und $\nu$ Bedingungen weg.

\msubsection{Spezialfälle des Maximumprinzips}

$H$ heißt regulär, wenn $\Omega$ konvex ist und $-\frac{\partial^2 H}{\partial^2 u}$ positiv definit ist.

\[ \forall t, y^*(t), \lambda(t), u\in \Omega \]

Dann ist das Maximierungsproblem im Maximumsprinzip konvex. Es hat eine eindeutige Lösung $u^*(t)$.

\[ h^*(t) := \max_{u\in\Omega} h(t,u) := \max_{u\in \Omega} H(t,y^*,u(t),\lambda(t)) \]

Dann ist $u^*(t)$ stetig und $h^*(t)$ stetig.

Beweis: Annahme $\exists \that$ mit $u(\that_-) \neq u(\that_+)$. Sei $h(\that, u(\that_-)) < h(\that, u(\that_+))$, $h(t,u) = H(t,y^*, u, \lambda)$ ist stetig in $u$. $\exists \eps: h(t,u(t)) < h(t,u(\that_+)) \forall t \in (\that-\eps,\that)$. Widerspruch zum Maximumprinzip. Auf ganz $(\that-\eps,\that)$ ist $u(t)$ nicht optimal.

Wenn $\Omega$ nicht konvex ist kann $u^*$ von einer Ecke in einer andere springen am Zeitpunkt $\that$. Wenn $h^*(t)$ stetig ist muss gelten: $h(u(\that_-),\that) = h(u(\that_+),\that)$. Das legt den Sprungzeitpunkt $\that$ fest.

$h(u,t)$ hängt nichtlinear von $u$ ab. Möglichkeiten:

\bitm
\item a) $u^* \in \circ{\Omega} \RA \frac{\partial h}{\partial u} (u^*(t),t) = 0$ und $\frac{\partial^2 h}{\partial^2 u} (u^*(t), t) \leq 0 \RA$ Wir können lokal eine Funktion $\hat u(t,y,\lambda) := \arg\max H(t,y^*,u,\lambda)$ finden.
\item b) $u^* \in \partial \Omega$ Dann gelten die KKT-Bedingungn, falls $\partial \Omega$ lokal durch Ungleichungen und Gleichungen beschrieben wird.
\eitm

Beim Übergang von a) nach b) und umgekehrt treten Schaltpunkte auf.

Beispiel: Raketenwagen $u \in [-1,1] \RA$ entweder $u \in \{-1,1\}$ oder $\frac{\partial h}{\partial u} = 0$.

$H$ ist linear in $u$.

\begin{align*}
H(t,y,u,\lambda) &= H_0(t,y,\lambda) + H_u (t,y,\lambda) u \\
\dot y(t,y,u) &= f_0 (t,y) + f_u(t,y)u \\
L &= L_0(t,y) + L_u(t,y)u \\
H_0 &= - L_0 + \lambda^T f_0 \\
H_u &= - L_u + \lambda^T f_u \\
\end{align*}

Sei \zb $u\in[-1,1]$.

\bitm
\item a) $H_u > 0 \RA u^* = +1$
\item b) $H_u < 0 \RA u^* = -1$
\eitm


Fall 1:

\[ u^*(t) = \begin{cases} +1 & t < \that \\ -1 & t > \that \end{cases} \]

Im Punkt $\that$ gilt ($H$ stetig in $u$):

\begin{align*}
H(\that_-) &= H_0 + H_u \cdot (+1) = H_0 + H_u \cdot (-1) = H(\that_+) \\
\RA H_u &= 0 = -L_u + \lambda^T f_u
\end{align*}

Heißt Schaltfunktion. Die Nullstelle legt den Schaltpunkt in $\that$ fest.

Fall 2: 

b) $H_u = 0$ gilt auf einem ganzen Intervall $[t_1,t_2]$

\[ \frac{\partial H}{\partial u} (t,y,(t),\lambda(t)) \equiv 0 \text{ auf } [t_1, t_2] \]

$u^*$ kann nicht durch das Maximumprinzip bestimmt werden. Dann gilt ("`Indexreduktion"'):

\bitm
\item $\l( \frac{\partial}{\partial t}\r)^T H_u (t,y(t),\lambda(t)) \equiv 0 $ $\forall i$ auf $[t_1,t_2]$
\item Dies hängt, wenn überhaupt, nur für $i = 2j$ von $u$ ab. Das kleinste solche $j$ heißt Ordnung.
\eitm

Sei \zb $j=1$: Singulärer Bogen der Ordnung $1$. $u$ wird festgelest durch $\ddot U_u \equiv 0$. $H_u(t_1) = 0 = \dot H_u(t_1)$ oder $H_u(t_1) = 0 = H_u(t_2) = 0$ legen $t_1$ und $t_2$ fest.








