\chaptr{11 Optimale Versuchsplanung}

\msection{11.1 Anwendungsbeispiel Urethan-Reaktion}

\begin{align*}
A + B & \stackrel{1}{\to} C \\
A + C &\stackrel{2,3}{\LRA} D \\
3 A & \stackrel{4}{\to} E
\end{align*}

\bitm
\item A: Isocyanat
\item B: Butanol
\item C: Urethan
\item D: Allophanat
\item E: Isozyanurat
\item L: Dimethylsulfonid
\eitm

TODO: Bild und so

DAE-Modell: $n_1, \cdots, n_6$: Molzahlen von $A,\cdots,L$

\begin{align*}
\dot n_3 &= V (r_1 - r_2 + r_3) \\
n_3(t_0) &= 0 \\
\dot n_4 &= V(r_2 - r_3) \\
n_4(t_0) &= 0 \\
\dot n_5 &= V r_4 \\
n_5(t_0) &= 0 \\
0 &= n_1 + n_3 + 2n_4 + 3n_5 - n_{1,0} - n_{1,0,Z_1} feed_1(t) \quad \text{ $Z_1$: Zulauf 1 } \\
0 &= n_2 + n_3 + n_4 - n_{2,0} - n_{2,0,Z_2} feed_2(t) \\
0 &= n_6 - n_{6,0} - n_{6,0,Z_1} feed_1(t) - n_{6,0,z_2} feed_2(t) 
\end{align*}

Algebraische Gleichungen ergeben sich aus Stöchioetrie.

\begin{align*}
\dot T &= \dd T (t) \\
T(t_0) &= T_0 \\
\dot{feed}_1 &= dfeed_1(t) \\
feed_1(t_0) &= 0 \\
\dot{feed}_2 &= dfeed_2(t) \\
feed_2(t_0) &= 0 \\
\end{align*}

Reaktionsgeschwindigkeiten:

\begin{align*}
r_1 &= k_1 \frac{n_1}V \frac{n_2}V \\
r_2 &= k_2 \frac{n_1}V \frac{n_3}V \\
r_3 &= k_3 \frac{n_4}V \\
r_4 &= k_4 \l( \frac{n_1}V \r)^2
\end{align*}

Geschwindigkeitskoeffizienten:

\begin{align*}
%k_i &= k_{ref} \exp l( -\frac{E_{ai}}R \l( \frac 1{T(t)} - \frac 1{T_{refi}} \r) \r) \quad i = 1,2,4 \\
\frac{k_2}{k_3} &= K_{c,2} \exp \l( - \frac{\Delta H_2}{R} \l( \frac{1}{T(t)} - \frac1{T_{g2}} \r) \r) \\
\end{align*}

Volumen:

\[ V = \sum\limits_{i=1}^6 \frac{M_i n_i}{\rho_i} \]

Das Modell enthält:

\bitm
\item 9 Zustände: $n_1, \cdots, n_6, T, feed_1, feed_2$
\item 8 unbekannte Parameter: $k_{ref1}, E_{a1}, k_{ref2}, E_{a2}, k_{ref4}, E_{a4}, K_{c2}, \Delta H_2$ 
\item 3 Steuerfunktionen $u(t)$: $\dd T(t), dfeed_1(t), dfeed_2(t)$
\item 8 Steuergrößen: $n_{1,0}, n_{2,0}, n_{6,0}, n_{1,0,Z_1}, n_{6,0, Z_1}, n_{2,0,Z_2},n_{6,0,Z_2}$
\item Restliche Größen sind Konstanten: $R, T_{ref}, M_i, \rho_i$
\eitm

Ziele:

\bitm
\item Schätzungen der Modellparameter auf experimentellen Daten.
\item Design der Experimente, so dass die geschätzten Parameter möglichst kleine statistische Unsicherheiten haben, also optimale Versuchsplanung.
\eitm

\msection{Statistische Analyse der Lösung des Parameterschätzproblems}

Modellgleichung (\obda ODE-RWP).

\begin{align*}
\dot y &= f(t,y(t),p,q,u(t)) \quad (11.1) \\
y(t_0) &= y_0(p,q) 
\end{align*}

\bitm
\item $p \in \R^{n_p}$: Parameter, werden durch die Natur bestimmt und sind erstmal unbekannt bzw. müssne geschätzt werden.
\item $q \in \R^{n_q}$: Steuergrößen und $u\colon [\tn, \tend] \to \R^{n_u}:$ Steuerfunktionen werden durch den Experimentator festgelegt.
\item $r(t_1, y(t_1), \cdots, t_k, y(t_k), p, q) = 0$: Randbedingungen (11.2)
\eitm

\msubsection{Modell für die Messungen}

\begin{align*}
\eta_i &= h_i(t_i, y(t_i), p, q) + \eps_i, \quad i = 1,\cdots,M \quad (11.3) \\
\eps_i & \sim \mathcal N \l( 0, \frac{\sigma_i^2}{w_i} \r), \cov (\eps_1, ¸eps_j) = 0, i\neq j \\
\end{align*}

$w_i \in \{0,1\}:$ Wird Messung $i$ tatsächlich durchgeführt? Für gegebene Daten: alle $w_i=1$. Für zu planende Experimente: Die Wahl von $w$ entscheidet über die Durchführung der Messung $i$.

\bitm
\item $w_1 = 1$: $\eps_i \sim \mathcal N(0, \sigma_i^2)$
\item $w_i = 0$: Die Messung $i$ hat unendliche Varianz.
\item $w_i \in [0,1]$, \zb $w_i = \frac 12$. "`Halbe Messung"' mit doppelter Varianz ($\to$ sinnvolle Relaxierung).
\eitm

Parameterschätzproblem:

\begin{align*}
\min_{p,y} \frac 12 &\sum\limits_{i=1}^M \frac{w_i}{\sigma_i^2} \l( \eta_i - h_i(t_i, y(t_i), p, q) \r)^2 \\
\text{s.\,t. } \dot y &= f(t,y,p,q,u) \\
y(t_0) &= y_0(p,q) \\
r(t_1, y(t_1), \cdots, t_k, y(t_k)) &= 0 \\
\text{mit } \xi :&=  (q,u,w) \text{ fest} \\
\end{align*}

Nach Parametrisierung der Dynamik, diehe Kapitel 3:

\begin{align*}
v &= (p,s) \\
\min_{v} \frac 12 \|F_1(v,\xi)\|_2^2 \quad (11.5) \\
\text{s.\,t. } 0 &= F_2(v, \xi) \\
F_1(v,\xi) &= \l( \frac{\sqrt{w_i}}{\sigma_i} \l( j_i(t_i,y(t_i, v, \xi), p, q) \r) \r)_{i=1,\cdots,M} \\
\end{align*}

$F_2(v,\xi)$ enthält die Randbedingungen und die zusätzlichen Gleichungsbeschränkungen aus der Parametriierung.

Jacobi-Matrix:

\begin{align*}
%\partial_{1ij} &= \frac{\partial F_{1i}{\partial v_j} = - \frac{\sqrt{w_i}}{\sigma_i} \l( \frac{\partial h_i}{\partial y} \frac{\partial y(t_i)}{\partial v_j} + \frac{\partial h_i}{\partial v_j} \r) \\
%\partial_{2ij} &= \frac{\partial F_{2i}{\partial v_j} = \frac{\partial F_{2i}}{\partial y} \frac{\partial y}{\partial v_j} + \frac{\partial F_{2i}{\partial v_j} \\
\end{align*}

$\frac{\partial y}{\partial v}$ sind Lösungen von Variationsdifferentialgleichungen nach Parametern und Anfangswerten.

Verallgemeinerte Inverse:

Lineares Problem:

\begin{align*}
\min &\frac 12 \|J_1 \Delta v + F_1\|_2^2 \\
\text{s.\,t. } 0 &= J_2 \Delta v + ?????
\end{align*}

Lösungsoperator:

\begin{align*}
J^\dagger &= \bpm I & 0 \epm \bpm J_1^T J_1 & J_2^T \\ J_2 & 0 \epm^{-1} \bpm J_1^T & 0 \\ 0 & I \epm \\
&\text{unbeschränkter Spezialfall: } J^\dagger &= (J_1^T J_1)^{-1} J_1^T \\
\Delta v &= -J^\dagger F, \quad F = \bpm F_1 \\ F_2 \epm
\end{align*}

Im Beispiel: Messwerte gegeben für Massenprozente von $A,C,D,E$ mit verschiedenen Varianzen an 5 Messzeitpunkten.

TODO: Bildchen

Ergebnis: Gesamtresiduum wird reduziert von $3\cdot 10^6$ auf $4\cdot 10^1$, die Kurven werden gut gefittet $\to$ "`Guter Fit"'

Aber: Ist das auch eine gute Schätzung der Parameter?

Es gilt: die Messdaten sind Zufallsgrößen. Der Lösungsalgorithmus des Parameterschätzproblems bildet die Messdaten auf den Parameterschätzer $\hat v_{ab}$. Also sind auch die geschätzten Parameter Zufallsgrößen.

TODO: mehr Bildchen wagen

Wir betrachten die Fehlerfortpflanzung für das linearisierte Parameterschätzproblem im Lösungspunkt $\hat v$. Dann gilt:

\[ \hat v \sim \mathcal N(v,C) \]

(ist die Kovarianzmatrix des Parameterschätzers).

Erwartungswert $E$:

\begin{align*}
E(\hat v) &= E(v_{wahr} + \Delta v) = E(\vwahr) + E(-J^\dagger F) \\
&= E(\vwahr) - J^\dagger E(F) = E(\vwahr) - J^\dagger E \bpm S^{-\frac 12} (\eta-h) \\ F_2 \epm \\
&= E(\vwahr) - J^\dagger 0 = \vwahr \\
\text{mit } S &= \diag \l( \frac{\sigma_i^2}{w_i}, i=1,\cdots,M \r) \\
\eta &= \bpm \eta_1 \\ \hdots \\ \eta_M \epm \\
h &= \bpm h_1 \\ \hdots \\ h_M \epm \\
C &= E(\hat v \hat v^T) = E( (\vwahr + \Delta v) (\vwahr + \Delta v)^T ) = E(\Delta v \Delta v^T) \\
&= E(J^\dagger F F^T (J^\dagger)^T \\
&= J^\dagger E \bpm F_1 F_1^T & F_1 F_2^T \\ F_2 F_1^T & F_2 F_2^T \epm (J^\dagger)^T \\
&= J^\dagger \bpm E(F_1 F_1^T) & 0 \\ 0 & 0 \epm (J^\dagger)^T \\
\text{mit } E(F_1 F_1^T) &= E( S^{-\frac 12} (\eta-h)(\eta-h)^T S^{-\frac 12}) \\
&= S^{-\frac 12} E(\eps \eps^T) S^{-\frac 12 } \\
&= S^{-\frac 12} S S^{-\frac 12} = I \\
\end{align*}

Also: $\hat v$ ist multinomialverteilt mit Erwartungswert $\vwahr$ und Kovarianzmatrix $C = J^\dagger \bpm I & 0 \\ 0 & 0 \epm (J^\dagger)^T$.

\msubsection{Bemerkung 11.1: Eigenschaften von C}

\bitm
\item a) $C$ existiert und ist eindeutig, wen CQ und PD erfüllt sind.
\item b) $C$ ist positiv semidefinit
\item c) $C$ hängt von Ableitungen des Parameterschätzproblems ab.
\item d) Wenn das Modell nichtlinear in $v$ ist hängt $C$ von den Parametern $\hat v$ ab.
\item e) $C$ hängt von dern Versuchsplanungsgrößen $\xi = (q,u,v)$ ab.
\item f) $C$ hängt nicht von den Messwerten ab.
\eitm

Unbeschränkter Fall:

\[ J^\dagger = (J_1^T J_1)^{-1} J_1^T \]

\[ C = (J_1^T J_1)^{-1}J_1^T J_1 (J_1^T J_1)^{-1} = (J_1^T J_1)^{-1} \]

Beschränkter Fall:

\[ J^\dagger = \bpm I & 0 \epm \bpm J_1^T J_1 & J_2^T \\ J_2 & 0 \epm \bpm J_1^T & 0 \\ 0 & I \epm \]

\[ C = \bpm I & 0 \epm \bpm J_1^T J_1 & J_2^T \\ J_2 & 0 \epm^{-1} \bpm J_1^T J_1 & 0 \\ 0 & 0 \epm \bpm J_1^T J_1 & J_2^T \\ J_2 & 0 \epm \bpm I \\ 0 \epm \quad (11.7) \]

\msubsection{Lemma 11.2}

Sei

\[ \bpm X & Y^T \\ Y & Z \epm := \bpm J_1^T J_1 & J_2^T \\ J_2 & 0 \epm^{-1} =:M^{-1} \]

Dann ist die Kovarianzmatrix $C=X$ und erfüllt

\begin{align*}
I &= J_1^T J_1 C + J_2^T Y \\
0 &= J_2 C
\end{align*}

Beweis: Multipliziere Matrix mit $M^{-1}$:

\begin{align*}
I &= J_1^T J_1 X + J_2^T Y \\
0 &= J_1^T J_1 Y + J_2^T Z \\
0 &= J_2 X \\
I &= J_2 Y^T \\
\end{align*}

Setze in (11.7) ein:

\begin{align*}
C &= \bpm I & 0 \epm M^{-1} \bpm J_1^T J_1 & 0 \\ 0 & 0 \epm M^{-1} \bpm I \\ 0 \epm \\
&= \bpm X & Y^T \epm \bpm J_1^T J_1 & 0 \\ 0 & 0 \epm \bpm X \\ Y \epm \\
&= X J_1^T J_1 X\\
&= X (\underbrace{J - J_2^T Y}_{=J_1^T J_1 X}) \\
&= X - (\underbrace{J_2 X}_{=0})^T Y \\
&= X \\
\end{align*}

\msubsection{Korollar 11.3}

\begin{align*}
C &= \bpm I & 0 \epm \bpm J_1^T J_1 J_2^T \\ J_2 & 0 \epm^{-1} \bpm I \\ 0 \epm \\
\end{align*}

\msection{Konfidenzgebiete}

Das $(100\cdot\alpha)\%$-Konvergenzgebiet ist das Gebiet, in dem die wahren aber unbekannten Parameter $\vwahr$ mit einer Sicherheit von $\alpha \in (0,1]$ liegen. Als Näherung betrachten wir das Konfidenzgebiet im Lösungspunkt $\hat v$ des Gauß-Newton-Verfahrens:

\begin{align*}
G(\alpha, \hat v) &= \{ v \in \R^n : F_2(x) = 0 \wedge \|F_1(v)\|_2^2 - \|F_1(\hat v)\|_2^2 \leq \gamma(\alpha)^2 \} \\
\text{mit } \gamma(\alpha)^2 &= \Chi_{n-n_2}^2 (a-\alpha)
\end{align*}

(Quantil der $\Chi^2$-Verteilung zum Wert $\alpha$ mit $n-n_2$ Freiheitsgraden, $v \in \R^n$, $F_1(v) \in \R^{n_1}$, $F_2(v) \in \R^{n_2}$).

\msubsection{Linearisierte Näherung}

\begin{align*}
G_L(\alpha) &= \{ v \in \R^n : F_2(\hat v) + J_2(\hat v) (v-\hat v) = 0, \\
& \|F_1(\hat v) + J_1(\hat v)(v-\hat v) \|_2^2 - \|F_1(\hat v)\|_2^2 \leq \gamma(\alpha)^2 \} \quad (11.10) \\
\end{align*}

\msubsection{Lemma 11.4}

Das linearisierte Konfidenzgebiet lässt sich in folgender Form schreiben:

\[ G_L(\alpha, \hat v) = \ov G_L(\alpha, \hat v) := \l\{ v \in \R^n, v - \hat v = J^\dagger(\hat v) \bpm \delta w \\ 0 \epm, \delta w \in \R^{n_1}, \|\delta w\|\leq \gamma(\alpha) \r\} \]

Beweis: siehe \zb Skript zu Numerik 3.

\msubsection{Korollar 11.5}

Sei $\theta_i = \gamma(\alpha) \sqrt{c_{ii} ( \hat v) }$, $i=1,\cdot,n$. Dann gilt: $G_L(\alpha, \hat v)$ wird durch einen Quader mit den Seitenlängen $2\theta_i$ eingeschlossen:

\[ G_L(\alpha,\hat v) \subseteq [ \hat v_1 - \theta_1, \hat v_1 + \theta_1] \times \cdots \times [\hat v_n - \theta_n, \hat v_n + \theta_n ] \quad (11.12)\]

Beweis:

\begin{align*}
|v_1-\hat v_1| &\leq \l\| \l( J^\dagger(\hat v) \bpm I \\ 0 \epm \r)^T e_i \r\| \|\delta w\|_2 \\
&= \l( e_i^TJ^\dagger(\hat v) \bpm I \\ 0 \epm \bpm I & 0 \epm J^\dagger (\hat v)^T e_i \r)^{\frac 12} \|\delta w\|_2 \\
&\leq \sqrt{c_{ii}(\hat v)} \gamma(\alpha) \\
\end{align*}

\msection{11.3 Das Versuchsplanungs- und Optimierungsproblem}

Wie kann die Güte des parameterschätzproblems verbessert werden?

Unbeschränkter Fall: $C = (J_1^T J_1)^{-1}$ Konfidenzintervall $\sqrt{c_{ii}}$. Mehr Messungen durchführen:

\begin{align*}
C_+ &= \l( \bpm J_1 \\ \hdots \\ J_1 \epm^T \bpm J_1 \\ \hdots \\ J_1 \epm \r)^{-1} = (m J_1^T J_1)^{-1} = \frac 1m (J_1^T J_1)^{-1} \\
\end{align*}

Die Konfidenzintervall werden nur um den Faktor $m^{-\frac 12}$ kleiner: $\sqrt{c_{+ii}} = \sqrt{m^{-1} c_{ii}} = m^{-\frac 12} \sqrt{c_{ii}}$.

Stattdessen: Entweder Experimente, die die statistische Unsicherheit der Parameterschätzung minimieren.

Versuchsplanungsgrößen: $\xi = (q,u,w)$:

\bitm
\item Steuergrößen $q \in \R^{n_q}$
\item Steuerfunktionen $u\colon [\tn, \tend] \to \R^{n_u}$
\item Auswahl der Messungen $w\in \{0,1\}^M$
\eitm

\[ w_i = \begin{cases} 1 & \text{ Messung wird durchgeführt} \\ 0 & \text{ Messung wird nicht durchgeführt} \end{cases} \]

Zielfuktion: Informationsfunktion auf $C$ ($C$ ist positiv semidefinit, symmetrisch).

Üblich sind:

\bitm
\item $\Phi(C) = \sqrt[n]{\det C}$ D-Kriterium 
\item $\Phi(C) = \frac 1n \mathrm{Spur} C$ A-Kriterium
\item $\Phi(C) = \|C\|_2 = \lambda_{max} (C)$ E-Kriterium
\item $\Phi(C) = \max_{i=1,\cdots,n} \sqrt{c_{ii}}$ U-Kriterium
\eitm

\msubsection{11.6}

Statt auf $C$ kann man die Kriterien auch auf $K^T C K$, die Projektion auf den $n_k$-dimensionalen Teilraum, $K \in \R^{n\times n_k}$ vollrangig, anwenden. Insbesondere für das D-Kriterium, wenn $C$ nicht positiv definit ist.

\msubsection{Bemerkung 11.7}

Im unbeschränkten Fall korrespondiert das D-Kriterium zum Volumen des Konfidenzellipsoids, das A-Kriterium zur Länge der durchschnittlichen Halbachsen, das E-Kriterium zur längsten Halbachsenlänge und das M-Kriterium zum größten Konfidenzintervall.



















