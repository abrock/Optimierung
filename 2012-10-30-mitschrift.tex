Bemerkung zu Ungleichsbedingungen: Wir betrachten PS-Probleme, bei denen man ein physikalisches Modell an experimentelle Daten fitten will. Typischerweise wird das Modell durch Gleichungen beschrieben: DGL-System, Anfangs- und Rand-Bedingungen. Ungleichungen werden dagegen vom Modellierer formuliert mit irgendwie willkürlichen Grenzen. Wenn wir ein PS-Problem mit Ungleichsbedingungen gelöst haben kann folgendes passieren:

\begin{itemize}
\item Im Lösungspunkt sind die Ungleichungen inaktiv, d.\,h. der Lösungspunkt erfüllt die echte Ungleichung $s(y(t_0), \cdots, y(t_k), p) > 0$. Dann können wir die Ungleichung weglassen.
\item Im Lösungspunkt sind Ungleichungen aktiv, aber die zugehörigen Lagrange-Multiplikatoren sind Null. Dann können wir die Ungleichung ebenfalls weglassen.
\item Eine Ungleichung ist aktiv mit Lagrange-Multiplikator ungleich Null, d.\,h. wenn man die Ungleichung weglässt wird die Zielfunktion besser, d.\,h. die Daten werden besser gefittet. Der Schätzer wird dann nicht nur durch die Daten sondern auch durch die vom Modellierer festgelegten Grenzen bestimmt. Das ist im allgemeinen physikalisch nicht sinnvoll.
\end{itemize}

Es kann sinnvoll sein, während der Algorithmus noch nicht terminiert ist, Grenzen zu fordern, um Auswertbarkeit des Modells zu gewährleisten. Im Lösungspunkt sollten diese Grenzen nicht aktiv sein. Wir behandeln bis auf weiteres Gleichungsbeschränkte PS-Probleme.

Allgemeine Problemformulierung:

\[ \min_{p,y} \frac 12 \sum \l( \frac{\eta_i - h_i(t_i,y(t_i),p)}{\sigma_i^2} \r)^2 \quad (2.14)\]

so dass

\begin{align*}
\dot y(t) &= f(t,y(t),p), \quad y(t_0) = y_0(p) \\
0 &= r(y(t_0), \cdots, y(t_k), p)
\end{align*}

oder mit DAE

\begin{align*}
\dot y(t) &= f(t,y,z,p), \quad y(t_0) = y_0(p) \\
0 &= g(t,y,z,p) 
\end{align*}

\section*{Lösungsmethoden}

\subsection*{Parametrisierung der Lösung des AWP durch Single Shooting, Multiple Shooting oder Kollokation}

Dadurch wird $(2.14)$ endlichdimensional:

\[ \min \frac 12 \|F_1(x)\|_2^2 \quad (2.15)\]

so dass $F_2(x) = 0$ mit $x \in \R^n$ geeignet.

\subsection*{Löse (2.15) mit verallgemeinertem Gauß-Newton-Verfahren}

\chapter*{3. Shooting-Verfahren und Kollokation}

\subsection*{3.1 Single-Shooting: Einfachschießverfahren}

% Bild: Messdaten mit Fehlerbalken, Lösung des AWP für einen Parametersatz

Vorgehensweise:
\begin{itemize}
\item Wähle Werte für die Parameter $p$ ("`Initial Guess"')
\item Wir lösen das AWP $\dot y = f(t,y,p)$, $y(t_0) = y_0(p)$ $(3.1)$ mit einem numerischen Verfahren und erhalten eine Darstellung der Lösung $y(t; t_0 y_0, p)$
\item Setze die Lösung an den Messzeitpunkten in die Modellantwortsfunktionen ein und berechne $F_{1,i}(p) = \sigma_i^{-1}(\eta_i - h_i(t_i, y(t_i; t_0, y_0, p), p)$, $i=1,\cdots,M$ (3.2). Setze die Lösung außerdem an den Randbedingungspunkten in die Randbedingung ein: $F_2(p) = r(y(t_0; t_0, y_0, p), \cdots, y(t_k; t_0, y_o, p), p)$ (3.3). Halte dazu den Integrator an den Punkten $t_i$ an oder benutze die fehlerkontrollierte kontinuierliche Ausgabe.
\item Das ergibt ein endlichdimensionales nichtlineares Ausgleichsproblem, nämlich $\min \tfrac 12 \|F_1(p)\|_2^2$ so dass $F_2(p) = 0$ (3.4).
\item Löse diese mit einer geeigneten Methode. Kriterien: (3.5)
\begin{itemize}
\item iterativ, da das Problem nichtlinear ist
\item sollte unzulässige Iterierte erlauben, d.\,h. Zwischenwerte, bei denen $F_2(p) \neq 0$
\item sollte für Least-Squares-Zielfunktion geeignet sein
\end{itemize}
\end{itemize}

Wir benutzen daher das verallgemeinerte Gauß-Newton-Verfahren. Dieses benötigt folgende Ableitungen:

\begin{align*}
J_1(p) :&= \frac \partial{\partial p} F_1(p) \\
J_2(p) :&= \frac \partial{\partial p} F_2(p)
\end{align*}

In jeder Iteration des Gauß-Newton-Verfahrens muss also das AWP gelöst und $F_1$ und $F_2$ ausgewertet werden.

\emph{Bemerkung 3.1} berechnung von $J_1$ und $J_2$.

\begin{align*}
J_1(p) :&= \frac{\partial F_1}{\partial p} (p) \\
&= \frac{\partial}{\partial p} \l( \frac{\eta_i - h_i(\cdots)}{\sigma_i} \r)_{i=1,\cdots,M} \\
&= \l( -\frac 1 \sigma_i \l( \frac{\partial h_i}{\partial_y} (\cdots) \frac{\partial y}{\partial p}(t_i; t_0, y_0, p) + \frac{\partial h_i}{\partial p} (t_i; t_0, y_0, p) \r) \r)_{i=1,\cdots,M} \quad (3.6)\\
J_2(p) &= \frac{\partial F_2}{\partial p} (p) \\
&= \sum_{i=0^k} \frac{\partial r}{\partial y_i} \frac{\partial y}{\partial p} (t_i; t_0, y_0, p) + \frac{\partial r}{\partial p} \quad (3.7)
\end{align*}

Dazu berechnet man $\tfrac{\partial y}{\partial p} =: G_p$ als Lösung der VDE

\[ \dot G_p = \frac{\partial f}{\partial y} G_p + \frac{\partial f}{\partial p} \quad (3.8)\]

sowie die Ableitungen

\[ \frac{\partial f}{\partial y}, \frac{partial f}{\partial p}, \frac{\partial h_i}{\partial y}, \frac{\partial h_i}{\partial p}, \frac{\partial r}{\partial y_i}, \frac{\partial r}{\partial p}\]

der Modellfunktion per Hand, durch numerische Differentiation, oder durch automatische Differentiation.

\section*{Algorithmus 3.2: Verallgemeinertes Gauß-Newton-Verfahren}

Zur Lösung von $\min_p \tfrac 12 \|F_1(p)\|_2^2$ s.\,t. $F_2(p) = 0$

\begin{itemize}
\item Start mit einer Startschätzung $p^0$, $k=0$ ("`Initial Guess"')
\item Solange ein Abbruchkriterium verletzt ist:
\begin{itemize}
\item Berechne $\delta p^k$ durch Lösung des linearisierten Ausgleichsproblems $\min \tfrac 12 \|F_1(p^k) + J_1(p^k) \delta p\|_2^2$ s.\,t. $F_2(p^k) + J_2(p^k) \delta p = 0$ (3.9)
\item Bestimme eine Schrittweite $\alpha^k$, z.\,B. durch Linesearch 
\item Iteriere $p^{k+1} = p^k + \alpha^k \delta p^k$ (3.10)
\end{itemize} 
\end{itemize}

% Bei Newton wird die ganze Zielfunktion linearisiert, bei Gauß-Newton wird innerhalb der Norm linearisiert

Mögliches Abbruchkriterium: $\|\delta p^k \| \leq \eps$. Mehr zu Gauß-Newton-Verfahren in Kapitel\,\,4.

\section*{Algorithmus 3.3: Single-Shooting Gauß-Newton}

\begin{itemize}
\item Start mit einer Startschätzung $p^0$, $k=0$
\item Solange ein Abbruchkriterium verletzt ist:
\begin{itemize}
\item Integriere das AWP (3.1) zusammen mit der VDE (3.8) für $p=p^k$
\item Halte an den Punkten $t_i$ an und werte $F_1(p^k)$ und $F_2(p^k)$ gemäß (3.2) und (3.3) aus. Berechne $J_1(p^k)$ und $J_2(p^k)$ gemäß (3.6) und (3.7)
\item Berechne $\delta p^k$ durch Lösen des linearen Ausgleichsproblems (3.9).
\item Berechne eine Schrittweite $\delta p^k$ und iteriere gemäß (3.10)
\end{itemize}
\end{itemize}

Implementierung: Praktische Aufgabe 1

Benötigte Bestandteile für die Implementierung:

\begin{itemize}
\item Integrator für AWP/VDE: entweder durch Integrator mit Interner Numerischer Integration ("`IND"'), siehe Kapitel 5. Oder durch Integrieren des Systems
\[\begin{pmatrix} \dot y \\ \dot G_p \end{pmatrix} = \begin{pmatrix} f \\ \frac{\partial f}{\partial y} G_p + \frac{\partial f}{\partial p}\end{pmatrix}, \quad y(t_0) = \begin{pmatrix} y_0(p) \\ \frac{\partial y_0}{\partial p}(p) \end{pmatrix} \]
\item Löser für lineare Ausgleichsprobleme $\min \tfrac 12 \|F_1 + J_1 \delta x\|_2^2$ s.\,t. $F_2 + J_2 \delta x = 0$. KKT-Bedingung:
\[ \exists \lambda: \begin{pmatrix} J_1^T J_1 & J_2^T \\ J_2 & 0 \end{pmatrix} \begin{pmatrix} \delta x \\ \lambda \end{pmatrix} = -\begin{pmatrix} J_1^T F_1 \\ F_2 \end{pmatrix} \]
\item Globalisierungs-Strategie: BT-Linesearch oder $\alpha^j = 1$
\end{itemize}






