\chaptr{6 Optimalitätsbedingungen für nichtlineare Optimierungsprobleme}

\msection{6.1 Allgemeine Problembeschreibung}

\[ \min_{x \in \R^n} f(x) \quad f \in C^2(\R^n, \R), \text{ "`Zielfunktion"' } (6.1) \]
\[ g(x) = 0 \quad g \in C^2(\R^n, \R^{m_1}), \text{ "`Gleichungsbedingungen"'} \]
\[ h(x) \leq 0 \quad h \in C^2(\R^n, \R^{m_2}), \text{ "`Ungleichungsbedingung"'} \]

Wir betrachten endlichdimensionale, kontinuierliche, glatte Probleme.

Wir unterscheiden

\bitm
\item unbeschränkte 
\item gleichungsbeschränkte
\item Gleichungs- und Ungleichungs-beschränkte Probleme
\eitm

\msubsection{Definition 6.1}

Ein Punkt $c \in \R^n$ heißt zulässiger Punkt des Problems (6.1), wenn er alle Nebenbedingungen erfüllt: $g(x) = 0$, $h(x \leq 0)$.

\msubsection{Definition 6.2}

Ein Punkt $x^* \in \R^n$ heißt lokales Minimum von (6.1) wenn er zulässig ist und wenn sein Zielfunktionswert kleiner oder gleich den Zielfunktionswerten aller zulässigen Punkte in einer Umgebung von $x^*$ ist.

\[ g(x^*) = 0 \wedge h(x^*) \leq 0 \wedge \exists \text{ Umgebung } U \text{ von } x^*: \forall x \in U \text{ mit } g(x) = 0, h(x) \leq 0 \text{ ist } f(x^*) \leq f(x) (6.2) \]

\msubsection{Definition 6.3}

Ein Punkt heißt globales Minimum, wenn er zulässig ist und sein Zielfunktionswert kleiner gleich den Zielfunktionswerten aller anderen zulässigen Punkte.

\[ x^* \text{ globales Minimum } \LRA g(x^*) = 0 \wedge h(x^*) \leq 0: \forall x \text{ mit } g(x) = 0, h(x) \leq 0 \text{ ist } f(x^*) \leq f(x) (6.3) \]

Die Bestimmung von globalen Optima heißt globale Optimierung. Wir beschränken uns in dieser Vorlesung auf die Berechnung lokaler Optima.

\msection{Definition 6.4}

Ein Minimum $x^*$ heißt strikt, wenn

\[ f(x^*) < f(x) \forall x \in U, x \neq x^*, x \text{ zulässig } (\text{lokal}) \]
\[ f(x^*) < f(x) \forall x \text{ zulässig } (\text{global}) \]

\msection{6.2 Optimalitätsbedingung im eindimensionalen Fall}

Lemma 6.5: Sei $f: (a,b) \subset \R \to \R$, $f \in C^2$. Dann gilt:

\bitm
\item Wenn $x^* \in (a,b)$ ein lokales Minimum von $f$ ist, ist $f'(x) = 0$ (notwendige Bedingung erster Ordnung) und $f''(x^*) \geq 0$ (notwendige Bedingung 2ter Ordnung)
\item Wenn $f'(x^*) = 0$ und $f''(x^*) > 0$, dann ist $x^*$ ein striktes lokales Minimum von $f$ (hinreichende Bedingung 2ter Ordnung).
\eitm

Beweis: Analysis 1

\msection{6.3 Unbeschränkter Fall}

Satz 6.6 (Notwendige Bedingungen erster und zweiter Ordnung)

Sei $f : \R^n \to \R$, $f \in C^2$. $x^*$ sei ein lokales Minimum von $f$.

\bitm
\item a) $\nabla f(x^*) = 0$ (6.4)
\item b) $\nabla^2 f(x^*) = 0$ (6.5)
\eitm

Beweis: Zurückführung auf den eindimensionalen Fall entlang beliebiger Kurven von Konkurrenten.

Sei $p \in \R^n$ beliebig. Die Funktion $\tilde f(t) := f(x^*+tp)$, $t\in \R$ hat in $x^*$ bzw. $t=0$ ein lokales Minimum. Also gilt nach Lemma 6.5 (eindimensionaler Fall):

\bitm
\item a) $0 = \tilde f'(0) = \frac{\partial}{\partial t} f(x^* + tp)|_{t=0} = f'(x^*) p = \nabla f(x^*)^T p \RA \nabla f(x^*) = 0$, da $p$ beliebig.
\item b) $0 \leq \tilde f''(0) = \frac{\partial^2}{\partial t^2} f(x^* + tp)|_{t=0} = p^T \nabla^2 f(x^*) p \RA \nabla^2 f(x^*)$ ist positiv semidefinit.
\eitm

\msubsection{Satz 6.7 (Heinreichende Bedingung)}

% ... Terme höherer Ordnung irgendwie wegdiskutieren
Sei $f: \R^n \to \R$, $f \in C^2$. Sei $x^* \in \R^n$ mit $\nabla f(x^*) = 0$ und $\nabla^2 f(x^*)$ positiv definit (6.6). Dann gilt: $x^*$ ist ein striktes lokales Minimum von $f$.

Beweis: Es existiert eine Umgebung $U$ von $x^*$ so dass die Hesse-Matrix $\nabla^2 f$ positiv definit für alle $x$ in der Umgebung $U$, da die Eigenwerte stetig von den Einträgen abhängen und die Einträge stetig von $x$ abhängen.
Entwicklung in eine Taylorreihe um $x^*$:

\begin{align*}
f(x) &= f(x^*) + \nabla f(x^*)^T (x-x^*) + \underbrace{\frac 12 (x-x^*)^T \nabla^2 f(\tilde x) (x-x^*)}_{>0} \text{ mit } \tilde x \in U \\
i\RA f(x) &> f(x^*) \\
\end{align*}

\msubsection{6.4 Gleichungsbeschränkter Fall}

Notation: $f: \R^n \to \R$, Ableitung $\frac{\partial f}{\partial x}f(x) = f'(x) = f_x(x) \in \R^n$ Zeilenvektor. Gradient: $\nabla f(x) = f_x(x)^T \in \R^n$ Spaltenvektor. Hessematrix $\nabla^2 f(x) = f_{xx}(x) = \frac{\partial}{\partial x} \nabla f(x) = \nabla \frac{\partial}{\partial x} f(x) \in \R^{n\times n}$ (symmetrische Matrix nach Satz von Schwarz).

$g: \R^n \to \R^m$, $n \leq m$, $\frac{\partial g}{\partial x} (x) = g_x(x) \in \R^{m\times n}$, $\nabla g(x) = g_x(x)^T \in \R^{n\times m}$. Menge aller zulässigen Punkte: $S := \{x : g(x) = 0\}$

\msubsection{Definition 6.8}

Ein Punkt $x^*$ heißt regulär, wenn er die Constraint Qualification (CQ) erfüllt: $\Rg(g_x(x^*)) = m$ (6.7).

\msubsection{Definition 6.9: Tangentialebene}

Die Menge $T(x^*) = \{ p : g_x(x^*) p = 0 \}$ (6.8) heißt Tangentialebene an $S$ in $x^* \in S$.

Laufe entlang der zulässigen Menge $S=\{x:g(x)=0\}$. Entweder wir schneiden die Höhenlinien von $f$, z.\,B. in $\hat x$. Dann erhöht oder erniedrigt ein kleiner Schritt den Zielfunktionswert, $\hat x$ ist also kein Optimum. Oder wir berühren eine Höhenlinie von $f$ tangential, hier in $x^*$. Dann kann $x^*$ ein Optimum sein.

Im lokalen Minimum gilt: Die Tangentialebene von $S$ und die Höhenlinien von $f$ sind parallel, also sind die Normalenvektoren parallel:

\[ -\nabla f(x^*) = \lambda \nabla g(x^*), \quad \lambda \in \R \]

Verallgemeinerung für $m$ Nebenbedingungen:

\begin{align*}
&-\nabla f(x^*) \text{ ist eine Linearbkombination der Gradienten der Nebenbedingungen:} \\
\exists \lambda \in \R^m : &-\nabla f(x^*) = \sum\limits_{i=1}^m \lambda_i \nabla g_i(x^*) = \nabla g(x^*) \lambda \quad (6.9) \\
\end{align*}

\msubsection{Definition 6.10}

Die Funktion $L\colon \R^n \times \R^m \to \R, (x,\lambda) \mapsto L(x,\lambda) := f(x) + \lambda^T g(x)$ (6.10) heißt Lagrangefunktion (Lagrangian) des beschränkten Optimierungsproblems $\min f(x) $ s.\,t. $g(x)=0$.

\msubsubsection{Bemerkung 6.11}

Gradient der Lagrangsfunktion:

\begin{align*}
\nabla_x L(x,\lambda) &= \nabla f(x) + \nabla g(x) \lambda \quad (6.11) \\
\nabla_\lambda L(x,\lambda) &= g(x) \\
\end{align*}

Hessematrix der Lagrangefunktion:

\[ \nabla^2 L(x,\lambda) = \bpm \nabla_{xx}^2 L(x,\lambda) & \nabla g(x) \\ \nabla g(x)^T & 0 \epm \quad (6.12) \]
\[ \text{mit } \nabla{xx}^2 L(x,\lambda) = \nabla_{xx}^2 f(x) + \underbrace{\nabla_{xx}^2 g(x) \lambda}_{\in \R^{n\times n}} \]

\msubsection{Satz 6.12 (Notwendige Bedingung erster Ordnung}

Betrachte das Problem $\min f(x)$ s.\,t. $g(x)=0$ mit $f\in C^2(\R^n,\R)$, $g \in C^2(\R^n, \R^m)$, $m \leq n$. Sei $x^*$ ein lokales Minimum, $x^*$ regulär. Dann existiert $\lambda \in \R^m$, so dass $\nabla_x L(x^*,\lambda) = \nabla f(x^*) + \nabla g(x^*) \lambda = 0$ und $\nabla_\lambda L(x^*, \lambda) = g(x^*) ) 9$ (6.13) bzw. $\nabla L(x.\lambda)=0$.

Beweis:

$x^*$ regulär $\RA g_x(x^*) = 0 \RA $ man kann $x$ zerlegen in $x = (y,z)$ so dass $g_y(x^*) \in \R^{m\times m}$ invertierbar. Führe die Fragestellung zurück auf den eindimensionalen Fall entlang Kurven von zulässigen Konkurrenzpunkten.

\begin{align*}
x &= \phi(t) \text{ mit } \phi(0) = x^*, \phi(1) = \ov x \neq x^* \text{ zulässiger Punkt} \\
\text{und } g(\phi(t)) &\equiv 0 \\
\phi(t) &= \bpm y(t) \\ z(t) \epm = \bpm y(t) \\ z^* + t(\ov z - z^*) \epm \\
0 &= \l. \frac{\partial}{\partial t} g(\phi(t))\r|_{t=0} = g_y(x^*) y'(0) + g_z(x^*) z'(x^*) \\
\RA \phi'(0) &= \bpm y'(o) \\ z'(0) \epm = \bpm -g_y(x^*)^{-1} g_z(x^*) \\ I \epm h \\
\end{align*}

$f(\phi(t))$ ist minimal in $t=0$. Daher ist $0 = \frac{\partial}{\partial t} f(\phi(t))|_{t=0} = f_x(x^*) \phi'(0) = (-f_y(x^*) g(x^*)^{-1} g_z(x^*) + f_z(x^*))h$.

Definiere $\lambda^T := -f_y(x^*)g_y(x^*)^{-1}$. Dann ist $f_y(x^*) + \lambda^T g_y(x^*) = 0$. $f_z(x^*) + \lambda^T g_z(x^*) = 0$ also ist $f_x(x^*) + \lambda^T g_x(x^*) = 0$ d.\,h. $\nabla L(x,\lambda)=0$.

$\nabla_\lambda L(x^*,\lambda) = g(x^*) = 0$, da $x^*$ zulässig.

\msubsection{Bemerkung 6.13}

Die Vektoren aus der Tangentialebene $T(x^*) = \ker g_x(x^*)$ haben die Gestalt:

\begin{align*}
p &= \bpm p_y \\ p_z \epm = \bpm -g_y(x^*)^{-1} g_z(x^*) \\ I \epm p_z \text{ denn } \\
0 &= g_x(x^*)p = g_y(x^*)p_y + g_z(x^*) p_z \\
\end{align*}

Alle zulässigen Kurven laufen also in der Tangentialebene in $x^*$ ein:

\begin{align*}
\phi(o) &= x^* \\
\phi'(0) &\in T(x^*)
\end{align*}

Die Spalten von

\[ \bpm -g_y(x^*)^{-1} g_z(x^*) \\ I \epm \]

bilden eine Basis von $T(x^*)$.

\msection{Satz 6.12}

$x^*$ optimal $\RA \nabla L(x^*,\lambda) = 0$ für ein $\lambda \in \R^m$.

\begin{align*}
0 &= \nabla_x L(x^*,\lambda) = \nabla f(x^*) + \nabla g(x^*) \lambda \\
0 &= \nabla_\lambda L(x^*, \lambda) = g(x^*) \\
\end{align*}

\msection{Definition 6.14}

Die notwendigen Bedingungen (6.13) heißen KKT-Bedingungen. Reguläre Punkte $x^* \in \R^n$, für die es $\lambda \in \R^n$ gibt, so dass $\nabla_{x,\lambda} L(x^*, \lambda) = 0$ heißen KKT-Punkte bzw. stationäre Punkte.

\msection{Bemerkung 6.15}

Stationäre Punkte können Minima aber auch Maxima oder Sattelpunkte sein.

\msection{Satz 6.16 (Notwendige Bedingungen zweiter Ordnung}

Seien die Voraussetzungen wie in Satz 6.12. Sei $x^*$ ein lokales Minimum, $x^*$ regulär. Dann ist die Hessematrix von $L$ nach $x$ und $x$ positid semidefinit auf $T(x^*) = \ker g_x(x^*)$, d.\,h.

\[ p^T \nabla_{xx}^2 L(x^*, \lambda) p \geq 0 \quad \forall p \in T(x^*) \]

Beweis: Weiter im Beweis von 6.12:

Es gilt:

\begin{align*}
0 &\leq \l. \frac{\partial^2}{\partial t^2} f(\phi(t)) \r|_{t=0} = \l. \frac{\partial}{\partial t} f_x(\phi(t)) \phi'(t) \r|_{t=0} \\
&= \l. \underbrace{ \phi'(t)^T f_xx(\phi(t)) \phi'(t)}_{\text{Krümmung von } f} \r|_{t=0} + \l. \underbrace{f_x(\phi(t)) \phi'(t)}_{\text{Krümmung von } \phi} \r|_{t=0} \\
\end{align*}

Außerdem gilt:

\begin{align*}
\frac{\partial^2}{\partial t^2} \lambda^T g(\phi(t)) &= 0 \text{ da } g(\phi(t)) \equiv 0 \\
\frac{\partial^2}{\partial t^2} \lambda^T g(\phi(t)) &= \phi'(t)^T (\lambda^T g_{xx} (\phi(t))) \phi'(t) + \lambda^T g_x(\phi(t)) \phi''(t) \\
\text{Also } 0 & \leq \phi'(0)^T (f_{xx} (x^*)^T + \lambda^T g_{xx}(x^*)) \phi'(0) + \underbrace{(f_{xx}(x^*) + \lambda^T g_x(x^*))}_{L_{x} (x^*, \lambda) = 0} \phi''(t) \\
&= \phi'(0)^T \nabla_{xx}^2 L(x^*, \lambda) \phi'(0) \\
\end{align*}

$\phi'(0)$ ist irgendein Vektor aus $T(x^*) \RA \nabla_{xx}^2 L(x^*, \lambda)$ ist positiv semidefinit auf $T(x^*)$.

\msection{Satz 6.17 (Hinreichende Bedingung)}

Seien die Voraussetzungen wie in Satz 6.12, aber $f,g\in C^3$. Wenn gilt: $\exists \lambda \in \R^n$, so dass $\nabla \lambda L(x^*, \lambda) = 0$ und $p^T \nabla_{xx}^2 L(x^*, \lambda) p > 0 \forall p \in T(x^*), p \neq 0$. Für einen regulären Punkt $x^*$, dann ist $x^*$ striktes lokales Minimum.

Beweis: Erfülle $x^*$ die hinreichende bedingung. Sei $\hat x \neq x^*$ zulässig und hinreichend nahe bei $x^*$. Betrachte die Kurve $\phi(t)$ mit $\phi(0) = x^*$, $\phi(1) = \hat x$,

\[ \phi(t) := \bpm \gamma(z(t)) \\ x^* + t(\hat z - z^*) \epm \]

so dass $g(\phi(t)) = 0$ in einer kleinen Umgebung von $x^*$

\begin{align*}
h :&= \hat z - z^*, \quad \|h\| \leq \eps_z \text{ Dann gilt } \\
\phi'(t) &= \bpm \gamma_z(z(t)) \\ I \epm h\\
\|\phi'(t)\| & \leq c_1 \|h\| \\
\phi'(0) &= \bpm \gamma_z (z^*) h \\ h \epm \in T(x^*) \leq c_1 \eps_z \\
\phi''(t) &= \bpm (\gamma_{zz} h)h \\ 0 \epm \\
\|\phi''(t)\| & \leq c_2 \|h\|^2 \leq c_2 \eps_z^2 \\
\phi'''(t) &= \bpm ((\gamma_{zzz} h)h)h \\ 0 \epm\\
\|\phi''(t)\| &\leq c_3 \|h\|^2 \leq c_3 \eps_3^3 \\
\end{align*}

Taylorentwicklung von $L(\phi(t),\lambda)$ um $t=0$:

% Da ist jetzt ein transponiert, pfffffft
% Ich weiß nicht, ob sich das wie ein Tensor transfomiert, halt 'ne dritte Ableitung

\begin{align*}
f(\hat x) - f(x^*) &= L(\hat x, \lambda) - L(x^*, \lambda) = L(\phi(t),\lambda) - L(\phi(0), \lambda) \\
&= \underbrace{\nabla_x L(\phi(0), \lambda)^T}_{=0} + \frac 12 \underbrace{ \phi'(0)^T \nabla_{xx}^2 L(\phi(0),\lambda) \phi'(0)}_{= (*) } + \frac 12 \underbrace{\nabla_x L(\phi(0), \lambda)^T}_{=0} \phi''(0) \\
&+ \frac 16 \l[ \underbrace{\phi'(\tilde t)^T ( \nabla_{xxx}^3 L(\phi(\tilde t),\lambda)^T \phi'(\tilde t)) \phi'(\tilde t)}_{\|\cdot\| \leq b \|h\|^2} + 3 \underbrace{\phi'(\tilde t)^T \nabla_{xx}^2 L(\phi(\tilde t), \lambda) \phi''(\tilde t)}_{\|\cdot\| \leq b \|h\|^3} + \underbrace{\nabla_x L(\phi(\tilde t), \lambda)^T \phi'''(\tilde t)}_{\|\cdot\| \leq b \|h\|^3} \r] \\
(*) &= h^T \bpm \gamma_z(z^*) \\ I \epm^T \nabla_{xx}^2 L(x^*, \lambda) \bpm \gamma_z(z^*) \\ I \epm h = h^T H h > 0 \\
\end{align*}

$H$ ist positiv definit und hat kleinsten Eigenwert $\lambda_{min} > 0$. Also ist $(*) \leq \lambda_{\min} \|h\|^2$,

\begin{align*}
f(\hat x) - f(x^*) & \geq \frac 12 \lambda_{min} \|h\|^2 - \frac 56 b \|h\|^3
\end{align*}

Für $\|h\| > 0$ genügend klein ist dies $>0$, also $f(\hat x) > f(x^*)$. Damit haben wir gezeigt, dass $x^*$ ein striktes lokales Minimum ist.

\msection{Satz 6.18: Stabilität}

Betrachte das gestörte Problem

\begin{align*}
&\min_x f(x, \tau) \quad (6.16) \\
&\text{s.\,t.} g(x,\tau) = 0 \\
\end{align*}

Gelten in $x^*$, $\tau = 0$ (ungestörter Lösungspunkt): Regularität, notwendige Bedingung erster Ordnung, hinreichende Bedingung zweiter Ordnung. Dann hängt die Lösung von (6.16) in einer Umgebung von $x^*$, $\tau=0$ stetig differenzierbar von $\tau$ ab. Die $x(\tau)$ sind alle strikte lokale Minima.

Beweis: Wende den Satz für implizite Funktionen auf das System $\nabla_{x,\lambda} L(x,\lambda,\tau) =: F(x,\lambda,\tau)=0$ an:

\begin{align*}
\l. \frac{\partial}{\partial (x,\lambda)} F(x,\lambda,\tau) \r|_{t=0} &= \bpm \nabla_{xx}^2 L(x^*,\lambda,0) & \nabla_x g(x^*, 0) \\ \nabla_x g(x^*, 0)^T & 0 \epm
\end{align*}

ist regulär weil $\nabla_x g(x^*, 0)$ vollen Rang hat und $\nabla_{xx}^2 L(x,\lambda,0)$ positiv definit auf $\ker \nabla_x g(x^*,0)^T$. Also hängen $x(\tau), \lambda(\tau)$ stetig differenzierbar von $\tau$ ab und sind stationäre Punkte in einer Umgebung von $\tau=0$. Die hinreichende Bedingung zweiter Ordnung gilt in einer Umgebung von $\tau=0$, also sind die $x(\tau)$ strikte lokale Minima. Die Lösung ist also stabil gegenüber kleiner Störungen der Zielfunktion und der Nebenbedingungen.

\msection{6.5 Probleme mit Ungleichungsbeschränkungen}

\msubsection{Definition 6.19}

Betrachte

\begin{align*}
\min f(x) & f \in C^2(\R^n, \R) \quad (6.18)\\
g(x)=0 & g \in C^2(\R^n, \R^{m_1}) \\
h(x) \leq 0 & h \in C^2(\R^n, \R^{m_2}) \\
\end{align*}

Die zulässige Menge von (6.18) ist $S := \{x: g(x)=0, h(x) \leq 0 \}$. Sei $J := \{1,\cdots,m_2\}$ die Indexmenge der Ungleichungen. Sei $x \in S$. Die Indexmenge der aktiven Ungleichungen ist $I(x) = \{i\in J : h_i(x) = 0\}$. Die Indexmenge der inaktiven Ungleichungen ist $I^\perp (x) = \{ i \in J: h_i(x) < 0$. $x^*$ ist regulär, wenn er die MFCQ oder die LICQ erfüllt.

\msubsection{Definition 6.20: MFCQ:}

Sei $x \in S$ zulässig und $I(x)$ die Indexmenge der in $x$ aktiven Ungleichungen. $x$ erfüllt die Mangasarian-Fromowitz-Constraint-Qualifications (MFCQ), wenn:

\bitm
\item die Gradienten $\nabla g_i(x=)$, $i=1,\cdots, m_1$ linear unabhängig sind.
\item ein Vektor $p \in \R^n$ existiert mit $\nabla g_i(x)^T p = 0$, $i=1,\cdots,m_1$ und $\nabla h_i(x)^T p < 0$, $i \in I(x)$.
\eitm

\msubsection{Definition 6.21 (LICQ)}

Sei $x\in S$ Indexmenge der in $x$ aktiven Ungleichungen. $x$ erfüllt die Linear Independence Constraint Qualifications, wenn die Gradienten $\nabla g_i(x)$, $i=1,\cdots,m_1$ und $\nabla h_i$, $i \in I(x)$ linear unabhängig sind, also die Menge $\{ \nabla g_i \} \cup \{ \nabla h_i \}$ linear unabhängig ist.

\msubsection{Satz 6.22}

LICQ $\RA$ MFCQ. Die Umkehrung gilt nicht. Beweis: Übungsaufgabe

Im Minimum gilt: $-\nabla f$ ist Linearkombination mit positiven Koeffizienten (konische Kombination) der $\nabla h_i$:

\[ \nabla f + \nabla h \mu = 0 \quad \mu \geq 0 \]

\msection{Satz 6.23 Notwendige Bedingungen}

\bitm
\item Notwendige Bedingung erster Ordnung: Sei $x^*$ ein lokales Minimum von (6.18), $x^*$ regulär. Dann existiert $\lambda \in \R^{m_1}$, $\mu \in \R^{m_2}$, $\mu_i \geq 0$ so dass $\nabla f(x^*) + \nabla g(x^*) \lambda + \nabla h(x^*) \mu = 0$, d.\,h. $\nabla_x L(x^*, \lambda, \mu) = 0$ für die Lagrangefunktion $L(x,\lambda, \mu) = f(x) + \lambda^T g(x) + \lambda^T h(x)$.

Außerdem gilt Komplementarität:
\[ \sum\limits_{i=1}^{m_2} \mu_i h_i(x^*) = \mu^T h(x^*) = 0 \]
gleichbedeutend:
\[ h_i(x^*) < 0 \RA \mu_i = 0, \quad \mu_i > 0 \RA h_i(x^*) = 0 \]
Für Punkte $x$, die die Komplementaritäts-Bedingung erfüllen ist 
\[ I^+(x) = \{ i \in I(x), \mu_i > 0 \} \]
die Indexmenge der strikt aktiven Ungleichungen.
\item Notwendige Bedingung zweiter Ordnung.
Sei $x^*$ ein lokales Minimum, $x^*$ regulär. Seien $\tilde h$ die Komponenten von $h$, die in $x^*$ aktiv  sind und $T(x^*) := \{ p : \nabla g(x^*)^T p = 0$, $\nabla \tilde h(x^*)^T p = 0 \}$ die Tangentialebene an $S$ in $x^*$. Dann gilt:
\[ p^T \nabla_{xx}^2 L(x^*, \lambda, p)p \geq 0 \quad \forall p \in T(x^*) (6.21) \]
Beweis: Schränke ein auf Umgebung $U$ von $x^*$, so dass $h_i(x) < 0 \forall x \in U \cap S \forall i \in I^\perp (x^*)$. Betrachte zulässige Konkurrenten $x \in U\cap S$, lasse nur solche zu, die auch $h_i(x) = 0$ erfüllen $\forall i \in I(x^*)$. Wegen Regularität: $|I(x^*)| + m_1 \leq n$
\[ \tilde S := \{ x: \tilde h(x) = 0 \} x \in U \cap S \cap \tilde S \]
$x^*$ ist ein lokales Minimum des gleichungsbeschränkten Problems bzgl. $S \cap \tilde S$.
Setze $\mu_i = 0$, $i \in I^\perp (x^*)$.
Aus dem Gleichungsbeschränkten Fall folgt dann (6.19).
$T(x^*)$ ist Tangentialebene des gleichungbeschränkten Problems. Es folgt aus dem gleischungsbeschränkten Fall die notwendige Bedingung zweiter Ordnung.
Komplementarität ist durch die Wahl von $\mu$ erfüllt: $\mu^T h(x^*) = 0$.
Zu zeigen bleibt: $\mu_i \geq 0$, $i \in I(x^*)$. Siehe folgendes Lemma:
\eitm

\msection{Lemma 6.24}

Sei $(x^*, \lambda, \mu)$ KKT-Punkt, $x^*$ regulär. Sei $\mu_i < 0$ für ein $i \in I(x^*)$. Dann existiert $\eps > 0$ und eine Kurve $\phi: [0,\eps] \to S$, so dass $\phi(0) = x^*$ und $f(\phi(t)) < f(\phi(0))$ für alle $t\in[0,\eps]$.

Beweis: betrachte Kurve $\phi$ mit $\dot \phi(0) = p$ mit

\begin{align*}
\bpm \nabla g^T \\ \nabla \tilde h_1^T \\ \vdots \\ \nabla \tilde h_{i-1}^T \\ \nabla \tilde h_i^T \\ \nabla \tilde h_{i+1} \\ \vdots \\ \nabla \tilde h^T \epm (x^*) p = \bpm 0 \\ 0 \\ \vdots \\ 0 \\ -\eps \\ 0 \\ \vdots \\ 0 \epm
\end{align*}

$p$ ist tengential zu allen Gleichungsbedingungen und aktiven Ungleichungsbedingungen außer $i$.

\begin{align*}
\tilde h_i(\phi(t)) &= \underbrace{\tilde h_i(x^*)}_{=0 \text{ aktiv}} + \underbrace{\nabla \tilde h_i (x^*)^T p}_{=-\eps \text{ nach Konstr.}} \cdot t + \underbrace{\mathcal O(\|p\|^2) t^2}_{\mathcal O(\eps^2)} \\
\leq 0 \text{ für } \eps &> 0 \text{ klein genug}
\end{align*}

Aus Regularität und Satz für implizite Funktionen folgt: Es gibt eine Kurve $\phi$ mit $\dot \phi(0) = p$ und 

\begin{align*}
\phi(t) &\in \{ x : g(x)=0, \tilde h_j(x) = 0, j\neq i \} \\
\l. \frac{\partial d}{\partial t} f(\phi(t))\r|_{t=0} &= \nabla f(x^*)^T p\\
&= - ( \lambda^T \nabla g(x^*)^T + \mu^T \nabla h(x^*)^T)p \\
&= - (-\mu_i \eps) \\
&= \mu_i\eps < 0
\end{align*}

\msection{Satz 6.25 (Hinreichende Bedingung)}

$(x^*, \lambda, \mu)$ erfülle die notwendige Bedingung erster Ordnung, $x^*$ sei regulär. Sei $p^T \nabla_{xx}^2 L(x^*, \lambda, \mu) p > 0 \forall p \in T^+(x^*), p\neq 0$ (6.22) mit $t^+(x^*) = \{ p: \nabla g(x^*) p = 0, \nabla h_i(x^*)^T p = 0 \forall i \in I^+(x^*) \}$, $I^+(x^*) = \{ i \in I(x^*), \mu_i > 0 \}$ (strikt aktive Ungleichung). Dann ist $x^*$ striktes lokales Minimum von (6.18). Es ist sogar auch striktes lokales Minimum von

\[ \min f(x) \text{ s.\,t. } g(x)=0, h_i(x) = 0, i \in I^+(x^*) \]

d.\,h. $h_i(x) = 0$ mit $\mu_i = 0$ und $h_i(x) < 0$ weggelassen.

Beweis:

Aus hinreichender Bedingung im gleichungsbeschränkten Fall folgt:

\bitm
\item $x^*$ ist striktes lokales Minimum von (6.23).
\item $x^*$ bleibt striktes lokales Minimum von (6.23) wenn man weitere Bedingungen, für die $x^*$ zulässig ist, hinzufügt; $h_i(x) \leq 0$, $i \in I^+(x^*)$. Für die weiteren zulässigen Punkte $x$ (in einer Umgebung von $x^*$ mit $h_i(x) < 0$, $i \in I^+(x^*)$ kann man eine Kurve $\phi(t)$ konstruieren, so dass $\l. \frac{\partial}{\partial t} f(\phi(t))\r|_{t=0}$, sofern alle $\mu_i > 0$ sind (analog zu oben). Dazu muss man den Satz für implizite Funktionen anwenden, dafür muss $\Rg \bpm \nabla g & \nabla \tilde h \epm^T = m_1 + |I(x^*)|$ in $x^*$ gelten (Regularität).
\eitm

\msection{Definition 6.26: Strikte Komplementarität}

In einem KKT-Punkt $(x^*, \lambda, \mu)$, $\mu \geq 0$ gilt strike Komplementarität, wenn:

\bitm
\item $h_i(x^*) = 0 \RA \mu_i > 0$
\item bzw. $\mu_i = 0 \RA h_i(x^*) < 0$
\item bzw. $I^+(x^*) = I(x^*)$
\item bzw. $h_i(x^*) = 0$ und $\mu_i = 0$ nicht gleichzeitig.
\eitm

Bemerkung: Wenn strikte Komplementarität gilt ist $T^+(x^*) = T(x^*)$. In der hinreichenden Bedingung ist dann diese Lücke zu notwendigen Bedingung geschlossen.

\msection{Satz 6.27: Stabilität}

 
Sei $(x^*, \lambda^*, \mu^*)$ ein KKT-Punkt, gelte strikte komplementarität, sei $x^*$ regulär und die hinreichende Bedingung zweiter Ordnung sei erfüllt. Dann ist das Minimierungsproblem

\[ \min f(x,\tau) \text{ s.\,t. } g(x,\tau)=0, h(x,\tau) \leq 0 \]

stabil gegen Störungen von $\tau$ um $0$ ($\tau=0$ ist das ursprüngliche Problem). D.\,h. es existieren Umgebungen $U$ von $(x^*, \lambda^*, \mu^*)$ und $V$ von $\tau = 0$ und eine stetig differenzierbare Abbildung $(x,\lambda, \mu)\colon V \to U$, $\tau \mapsto (x(\tau), \lambda(\tau), \mu(\tau))$ mit $x(0) = x^*$, $\lambda(0) = \lambda^*$, $\mu(0) = \mu^*$ und $x(\tau)$ ist striktes lokales Minimum .

Beweis:
Betrachte

\begin{align*}
\nabla f(x,\tau) + \nabla g(x,\tau) \lambda + \nabla \tilde h(x,\tau) \mu &= 0 \\
g(x,\tau) &= 0 \\
\tilde h(x,\tau) &= 0 \\
\end{align*}

Jacobi-Matrix bezüglich $(x,\lambda, \mu)$ in $\tau = 0$:

% so, bischen weihnachtlich, kommt noch 'ne andere Farbe

\[ \bpm \nabla_{xx}^2 L(x^*, \lambda^*, \mu^*) & \nabla g(x^*) & \nabla \tilde h(x^*) \\ \nabla g(x^*)^T & 0 & 0 \\ \nabla \tilde h(x^*)^T & 0 & 0 \epm \]

ist regulär.

Mache $V$ möglicherweise kleiner, so dass

\bitm
\item $\mu_i(\tau) > 0$ ($\mu_i(0) > 0$ wegen strikter Komplementarität)
\item $h_i(x(\tau)) < 0$ für $i \notin I(x^*)$. Setze deren $\mu_i(\tau)=0$
\item $\bpm \nabla g^T \\ \nabla \tilde h^T \epm (x(\tau))$ vollranging
\item $\nabla_{xx}^2 L(x(\tau), \lambda(\tau), \mu(\tau))$ positiv definit auf $T(x(\tau)) \RA$ hinreichende Bedingung für $x(\tau), \lambda(\tau), \mu(\tau)$, $\tau \in V$
\eitm

% Ich darf ja nicht eine Farbe ganz abnutzen, nachher kommt die Mathematik für Biologen Übung und dann können die nicht mehr ihre bunten Bildchen an die Tafel malen, wenn es kein Rot mehr gibt.

Bemerkung: Wir haben in den Beweisen LICQ als Regularitätsbedingung benutzt. MFCQ kann auch verwendet werden, Beweise siehe z.\,B. Geiger Kanzov: Nichtlineare Optimierung, Springer.











