
\chaptr{9. Direkter Ansatz der optimalen Steuerung}

Betrachte \obda Probleme mit fester Anfangs- und Endzeit und Lagrange-Zielfunktional.

Schwierigkeit: $u\colon [t_0,\tend] \to \R^{n_u}$ ist Element eines unendlichdimensionales Funktionenraumes. Idee: Drücke $u$ durch endlich viele Variablen $q_i$ aus, zum Beispiel durch lokale Ansatzfunktionen:
Wähle Gitter: $t_0 = \tau_0 < \tau_1 < \cdots < \tau_m = \tend$. Ersetze $u(t)$ auf $t \in [\tau_i, \tau_{i+1})$ durch eine lokale Ansatzfunktion $u(t) = \phi_i(t,q_i)$, $t \in [\tau_i, \tau_{i+1})$, $i = 0, \cdots, m-1$ mit endlich vielen Parametrisierungsvariablen $q_i \in \R^{n_{q_i}}$.

Typische Beispiele: Stückweise Polynome, stückweise linear, stückweise linear und stetig.

Stetigkeitsbedinung: $\phi_i (t_{i+1}, q_i) = \phi_{i+1} (t_{i+1}, q_{i+1})$

\msubsection{Bemerkung 9.1}

Stückweise lineare stetige Steuerfunktionen können auch durch Integration von stückweise konstanten Steuerfunktionen gebildet werden. Zusätzliche DGL: $\dot y_{n+1} = u$ hat als Lösung eine stückweise lineare Funktion.

\msubsection{Bemerkung 9.2}

\bitm
\item Verschiedene Komponenten von u können verschieden behandelt werden.
\item Wenn man Ansatzfunktionen mit lokalem Träger (insbesondere keine Spline-Funktionen) verwendet, dann werden die Größen $q_i$ nur auf dem intervall $[\tau_i, \tau_{i+1})$ vorkommen $\RA$ Entkopplung $\RA$ Blockstruktur der Hessematrix, siehe unten.
\item Prinzipiell können im direkten Ansatz aber beliebige Parametrisierungen mit endlich vielen Freiheitsgraden verwendet werden.
\eitm

Damit haben wir endlich viele Variablen erhalten:

\bitm
\item $q$: Zeitunabhängige Steuerungen
\item $q_0, \cdots, q_{m-1}$ Parametrisierungsvariablen auf $[\tau_i, \tau_{i+1})$
\eitm

Fasse zusammen: $\ov q := (q, q_0, \cdots, q_{m-1}) \in \R^{n_q}$.

\msubsection{Umformulierung der Steuerungsbeschränkungen}

Beschränkungen an $u(t)$ gehen über in Beschränkungen an $q_i$, \zB $u_l \leq u(t) \leq u_u$ geht über in $q_l \leq q_i \leq u_u$, $i=0,\cdots,m-1$ bei stückweise konstanter Parametrisierung. Bei stückweise linearer Parametrisierung: am Anfang und Ende der Intervalle: $u_l\leq q_{i0} \leq u_u$, $u_l \leq q_{i0} + (\tau_{i+1} - \tau_i) q_{i1} \leq u_u$.

Auch die Zustandsbeschränkungen $b(t,y(t),q,u(t)) \geq 0$ sind unendlichdimensional. Behandlung: Punktweise Auswertung auf einem Gitter $t_0 \leq t_1 < \cdots < t_k \leq \tend$, $b(t_j, y(t_j), q, \phi_{i(j)} (t_j, q_{i(j)})) \geq 0$, $j=1,\cdots,k$, $i(j)$ so, dass $t_j \in [\tau_{i(j)}, \tau_{i(j)+1})$. Alternative: Penalty-Term in der Zielfunktion, kontinuierliche Behandlung.

Insgesamt erhalten wir:

\begin{align*}
\min_{q,\ov q} &\sum\limits_{i=0}^{m-1} \int\limits_{\tau_i}^{\tau_{i+1}} L(t, y(t), q, \phi_i(t, q_i)) \dd t = \sum\limits_{i=0}^{m-1} \int\limits_{\tau_i}^{\tau_{i+1}} \ov L(t, y(t), \ov q) \dd t \quad (9.3) \\
\text{\st } \quad \dot y &= f(t,y(t), q, \phi_i(t,q_i) \quad t \in [\tau_i, \tau_{i+1}), \quad i = 0, \cdots, m-1 \\
&= \ov f(t,y(t),\ov q) \\
\text{und Randbedingungen: } y(t_0) &= y_0(q) \\
r(t_0, y(t_0), \cdots, \tend, y(\tend), q) &= 0 \\
\text{Punktweise Bedingungen: } &b(t_j, y(t_j), q, \phi_{i(j)} (t_j, q_{i(j)})) \quad j=1,\cdots, k \\
&= \ov b(t_j, y(t_j), \ov q) \geq 0 \\
\text{Steuerungsbeschränkungen: } q_l & \leq \ov q \leq q_u \\
b_l & \leq A\ov q \leq b_u \quad \text{ mit geeigneten } q_l, q_u, b_l, b_u, A 
\end{align*}

zu tun bleibt: Parametrisierung der Zustandsvariablen.

Beispiel: Bimolekulare Katalyse, siehe auch Kapitel 2.

\begin{align*}
A+B & \to C \\
\intertext{Molzahlen } n_1, n_2, n_3, t \in [0,10] \\
k &= k_1 + k_2 \\
&= f_1 \exp \l( -\frac{E_1}{RT} \r) +c_{kat} \exp \l( -\lambda t \r) f_2 \exp \l( - \frac{E_2}{RT} \r)\\
\dot n_1 &= - V k \frac{n_1}V \frac{n_2}V, n_1(0) = n_{10} \\
\dot n_2 &= - V k \frac{n_1}V \frac{n_2}V, n_2(0) = n_{20} \\
\dot n_3 &= V k \frac{n_1}V \frac{n_2}V, n_3(0) = 0 \\
\end{align*}

Der Prozessverlauf wird bestimmt durch:

\bitm
\item Parameter $f_1, E_1, f_2, E_2, \lambda$, jetzt Konstanten.
\item Steuergrößen: \[ q = \begin{cases} V: & \text{ Volumen} \\
n_{10}: & \text{ Anfangsmolzahl von A} \\
n_{20}: & \text{ Anfangsmolzahl von B} \\
c_{kat}: & \text{ Katalysatorkonzentration} \end{cases} \]
\eitm

Mögliche Zielfunktion: Maximiere die Molzahl des Reaktionsproduktes: $\min -n_3(\tend)$ (Meyer-Zielfunktion). Parametrisierung der Steuerfunktion: Stückweise linear und stetig auf 3 Intervallen:

\begin{align*}
t \in [0,2): T(t) &= q_5 + t q_6 \\
t \in [2,8): T(t) &= q_7 + (t-2) q_8 \\
t \in [8,10]: T(t) &= q_9 + (t-8)q_{10} \\
\end{align*}

Nebenbedingungen:

\begin{align*}
q_6 &= 0 \\
q_7 &= q_5 \\
q_7 + 6 q_8 &= q_9 \\
q_{10} &= 0
\end{align*}

Grenzen:

\begin{align*}
q_5 &\in [293, 393] \\
q_9 &\in [293, 393] \\
\end{align*}

\msection{9.2 Direktes Single-Shooting}

Wähle Werte für die Anfangswerte $y(t_0) = s_0$ (oder konstant oder $q_j$) und für die Steuerungen $\ov q = (q,q_0,\cdots,q_{m-1})$. Löse Anfangswertproblem $\dot y 0 f(t,y(t),q,\phi_0(t,q_0))$, $t\in[\tau_0,\tau_1)$, $y(t_0) = s_0$. $\dot e = L(t,y(t),q,\phi_0(t,q_0))$, $t \in [\tau_0, \tau_1)$, $e(t_0) = 0$ (9.4) und sukzessive $\dot y = f(t,y,q,\phi_i(t,q_i))$, $\dot e = L(t,y,q,\phi_i(t,q_i))$, $t\in[\tau_i, \tau_{i+1})$. Liefert Lösung $y(t,s_0,q,q_0,q_1,\cdots,q_{m-1})$ und Zielfunktionswert $e(t,s_0,q,q_0,q_1,\cdots,q_{m-1})$. Interessant ist nur $e(\tend,s_0,q,q_0,\cdots,q_{m-1})$

Bemerkung: $y(t,s_0,q,q_0,q_{m-1})$ hängt nur von den $q_i$, $i=0,\cdots,j$ ab für $t \leq \tau_{j+1}$. Damit ergibt sich ein endlichdimensionales Optimierungsproblem in den Variablen $X=(s_0,q,q_0,\cdots,q_{m-1})$ (9.3)

\begin{align*}
\min_x F_1(x) &= e(\tend, x) \\
\text{\st } \quad F_2(x) &= r(t_0, y(t_0, x), \cdots, \tend, y(\tend, x), q) =: \ov r(x) = 0, \quad j=1,\cdots,k\\
F_3(x) & \geq 0 \quad \text{ umfasst alle Ungleichungen: } \ov b(t_j,y(t_j,x),x) \geq 0, \quad q_l \leq \ov q \leq q_u, \\
b_l & \leq A \ov q \leq b_u \\
\end{align*}

Zur Auswertung von Zielfunktion und Nebenbedingungen von (9.6) sind jeweils die Anfangswertprobleme (9.4) zu lösen. Das Optimierungsproblem:

\begin{align*}
\min_x F_1(x) \text{ s.\,t. } F_2(x) = 0 \text{ s.\,t. } F_3(x) \geq 0 
\end{align*}

ist ein NLP und kann mit NLP-Methoden gelöst werden, am besten mit SQP-Verfahren, siehe Kapitel 7.

\msection{9.3 Direktes Multiple Shooting}

Problem beim Single-Shooting: Mit schlechten Startwerten starten wir weit weg von der Lösung und können Schierigkeiten bei der Konvergenz des SQP-Verfahrens bekommen. Wenn wir über die Steuergrößen $(q, q_0, \cdots,q_{m-1})$ keine Informationen haben, bekommen wir ggf. Trajektorien $y(t)$, die sehr weit von der optimalen Lösung entfernt sind oder evtl. nicht auf dem ganzen Intervall $[t_0, \tend]$ existieren. Aber: Der ungefähre Verlauf von $y(t)$ ist oft durchaus bekannt.

\msection{Mehrzielmethode / Multiple Shooting}

\bitm
\item Wähle ein Gitter der Mehrzielknoten $t_0 = \tau_0 < \tau_1 < \cdots < \tau_m = \tend$ (9.7). Zur Vereinfachung: gleiches Gitter wie für Parametrisierung der Steuerungen , aber das ist nicht notwendig.
\item Wähle Werte für die Anfangswerte der Zustände in den Mehrzielknoten: $y(\tau_i) = s_i$, $i=0,\cdots,m-1$. Hier: Möglichkeit, Vorinformationen einzubringen.
\item Wähle Werte für die parametrisierten Steuerungen $q_i$, $i=0,\cdots,m-1$. \obda diskutieren wir die nicht zeitabhängigen Steuergrößen $q$ weg. Dafür gibt es zwei Möglichkeiten:
\bitm
\item Formuliere sie als stückweise konstant und stetig parametrisierte Steuerfunktion.
\item Formuliere zusätzliche Zustände.
\[ \bpm y_{n_y +1} \\ \vdots \\ y_{n_y +n_q} \epm \]
mit $\dot y_{n_y +j} \equiv 0$.
\eitm
Der Integrator sollte diese Zustände nicht mitintegrieren und keine VDE dafür lösen.
\item Variablen sind also $X = (s_0, q_0, s_1, q_1, \cdots, s_{m-1}, q_{m-1})$ (9.8) 
\item Löse Anfangswertproblem
\begin{align*}
\dot y &= f(t,y,\phi_i(t,q_i)) = \ov f(t,y,q_i) \quad (9.9) \
y(\tau_i) &= s_1 \\
\dot e &= L(t,y,\phi_i(t,q_i)) = \ov L(t,y,q_i) \\
e(\tau_i) &= 0 \quad i = 0,1,\cdots,m-1
\end{align*}
Es wird eine unstetige Trajektorie, bestehend auf Teiltrajektorien $y(t,\tau_i, s_i, q_i)$, $t \in [\tau_i,\tau_{i+1})$.
\item Die Gesamttrajektorie muss stetig gemacht werden durch Anschlussbedingungen:
\begin{align*}
h_i(s_i,q_i,s_{i+1}) :&= y(\tau_{i+1}, \tau_i, s_i, q_i) - s_{i+1} = 0 \quad i=0,\cdots, m-2 \
\end{align*}
Anfangsbedingung bei $t_0 = \tau_0$: $y(t_0) = s_0$. Das sind insgesamt $((m-1)+1)\cdot n_y = m \cdot n_y$ Bedingungen für die $m \cdot n_y$ Zustandsvariablen.
\item Zielfunktional:
\begin{align*}
&\min \sum\limits_{i=0}^{m-1} e(\tau_{i+1}, \tau_i, 0, q_i)  \quad (9.11)\\
&= F_1(X)
\end{align*}
\item Nebenbedingungen: Setze $y(t, \tau_i, s_i, q_i)$ ein in Randbedingungen: 
\[ r(\tau_0, s_0, \tau_1, s_1, \cdots, \tau_{m-1}, s_{m-1}, \tend, y(\tend, \tau_{m-1}, s_{m-1}, q_i)) = 0 \]
Besser: separierbare Randbedingungen:
\[ \sum\limits_{i=0}^{m-1} r_i(\tau_i, s_i) + r_m(\tend, y(\tend, \tau_{m-1}, s_{m-1}, q_{m-1})) = 0  \quad (9.12)\]
verwende diese Variante
\bitm
\item In Zustandsbeschränkung:
\[ b(\tau_i, s_i, q_i) \geq 0, \quad i=0,\cdots,m-1 \]
Zur Vereinfachung: gleiches Gitter (nicht notwendig).
\eitm
\eitm

Weitere Nebenbedingungen: Anschlussbedingungen, Steuerungsbeschränkungen

\begin{align*}
q_{lo} & \leq q \leq q_{up} \\
v_{lo} &\leq A q \leq b_{up} \
\end{align*}

Zur Auswertung von Zielfunktion und Nebenbedingungen werden als Anfangswertprobleme auf den Mehrzielintervallen gelöst: $t \in [\tau_i, \tau_{i+1} )$. Diese Integrationen sind entkoppelt und können unabhängig voneinander parallel durchgeführt werden, dadurch kann das Problem auf modernen Mehrprozessor-Rechnern schneller als mit Single-Shooting gelöst werden. Zielfunktion und Nebenbedingungen sind separierbar, sie bestehen aus Beiträgen der einzelnen Mehrzielintervalle, die höchstens linear (hier als Summanden) gekoppelt sind.

Zielfunktion:

\[ \sum\limits_{i=0}^{m-1} e(\tau_{i+1}, \tau_i, s_i, q_i) \]

Anschlussbedingungen:

\[ y(\tau_{i+1}, \tau_i, s_i, q_i) - s_{i+1} = 0 \]

Randbedingungen:

\[ \sum\limits_{i=0}^{m-1} r_i(\tau_i, s_i) + r_m(\tend, y(\tend, \tau_{m-1}, s_{m-1}, q_{m-1})) = 0 \]

Resultierendes Optimierungsproblem:

\begin{align*}
\min_x &F_1(x) \quad (9.13) \\
F_2(x) &= 0 \quad \text{ Anschlussbedingungen, Randbedingungen} \\
F_3(x) & \geq 0 \quad \text{ Zustandsbeschränkungen, Steuerungsbeschränkungen}\\
\end{align*}

Das Optimierungsproblem (9.13) hat die gleiche Lösung wie das aus der Single-Shooting-parametrisierung hervorgehende (9.6). Einerseits hat (9.13) viel mehr Variablen: $m n_y + m n_u f$ gegenüber $n_y + m n_u f$, andererseits hat es auch viel mehr Struktur:

\bitm
\item 1): Die Jacobimatrix $A$ der Nebenbedingungen ist dün besetzt.
\item 2): Die Hesse-Matrix der Lagrange-Funktion ist Block-diagonal.
\eitm

Zu 1):

\begin{align*}
& \text{Anschlussbedingungen:} \\
\frac{\partial h_i}{\partial x} &= \l( 0,\cdots,0, \frac{\partial h_i}{\partial s_i}, \frac{\partial h_i}{\partial q_i}, \frac{\partial h_i}{\partial s_{i+1}}, 0, \cdots, 0 \r) \\
&= \l( 0,\cdots,0, \frac{\partial y}{\partial s_i} (\tau_{i+1}, \tau_i, s_i, q_i), \frac{\partial y}{\partial q_i} (\tau_{i+1}, \tau_i, s_i, q_i), -I, 0, \cdots, 0 \r) \\
&= (0,\cdots, 0, G_i, \Gamma_i, -I, 0,\cdots,0) \\
& \text{Randbedingungen:} \\
\frac{\partial r}{\partial x} &= \l( \frac{\partial r_0}{\partial s_0}, 0, \frac{\partial r_1}{\partial s_1}, 0, \cdots, \frac{\partial r_{m-1}}{\partial s_{m-1}} + \frac{\partial r_m}{\partial y}{\partial y}{\partial s_{m-1}} (\tend, \tau_{m-1}, s_{m-1}, q_{m-1}), \frac{\partial r_m}{\partial y} \frac{\partial y}{\partial q_{m-1}} (\tend, \tau_{m-1}, s_{m-1}, q_{m-1}) \r) \\
\end{align*}

$\nabla F_3(x)^T: $ Zustandsbeschränkungen:

\begin{align*}
\frac{\partial}{\partial x} b_i &= \l( 0,\cdots, 0, \frac{\partial b_i}{\partial s_i}, \frac{\partial b_i}{\partial q_i}, 0, \cdots, 0 \r) 
\end{align*}

Zu 2): 

\begin{align*}
\mathcal L(x,\lambda) &= \sum\limits_{i=0}^{m-1} e(\tau_{i+1}, \tau_i, 0, q_i) + \sum\limits_{i=0}^{m-1} \lambda_{Ai}^T h_i (s_i, q_i, s_{i+1}) + \lambda_R^T \l( \sum\limits_{i=0}^{m-1} r_i + r_m \r) \\
&+ \sum\limits_{i=0}^{m-1} \lambda_{Bi}^T b_i(\tau_i, s_i, q_i) + \lambda_s^T (\text{Steuerbeschr.})
\end{align*}

Offenbar:

\begin{align*}
\frac{\partial^2 L}{\partial \bpm s_i \\ q_i \epm \partial \bpm s_j \\ q_j \epm } &= 0 \quad \text{ für } i\neq j
\end{align*}

$\RA$ Blockdiagonalität.

Wir berechnen $H$ durch Low-Rank-Updates, z.\,B. BFGS, siehe Kapitel 7. Statt ganz $H$ aufzudatieren Rang 2 pro Schritt, kann man auch einzelne Blöcke mit BFGS aufdatieren, Rang 2 pro Block, Rang $2\cdot m$ insgesamt. Die Hessematrix wird viel besser approximiert, dadurch verbessert sich das Konvergenzverhalten des SQP-Verfahrens drastisch, siehe Satz von Dennis-More.

Bei der QP-Lösung können/müssen die Strukturen in der Mehrzielmatrix ausgewertet werden:

\[ \bpm H & A^t \\ A & 0 \epm \]

\msubsection{Kondensieralgorithmus (Plitt 1981, Bock, Plitt 1984}

wird alle $\Delta s_i$, $i=1,\cdots,m-1$ aus QP raus

\[ \Delta s_{i+1} = G_i s_i + \Gamma_i \Delta q_i + h_i \quad \mathcal O(n_y^2) \]

Neues QP in den Variablen $\Delta s_0$, $\Delta q_0, \cdots, \Delta q_{m-1}$:

\[ \bpm H_{voll} & B^T \\ B & 0 \epm \]

Matrix hat keine "`schöne"' Struktur mehr, Lösen in $\mathcal O(n^3)$

Problem wird reduziert auf Dimension Anzal $s_0 + $ Anzahl $q_0,\cdots,q_{m-1} \RA$ gleiche Problemgröße wie bei Single Shooting.

\msubsection{Umsortieren der Variablen}

$(s_0, q_0, \lambda_0, s_1, q_1, \lambda_1, \cdots ) \RA$ Blockstruktur der Gesamtmatrix, symmetrisch aber nicht positiv definit. $\RA$ Benutze iterative oder Sparse-Löser. Aufwand: $\mathcal O(m)$.

\msubsection{Vorteile von Multiple Shoting gegenüber Single Shooting}

\bitm
\item Es können Informationen über den Verlauf der Trajektorien benutzt werden. Dadurch startet man ggf. näher bei der Lösung und im Einzugsbereich der lokalen Konvergenz des SQP-Verfahrens.
\item Durch die kürzeren Integrationsintervalle wird die Nichtlinearität des Problems reduziert (Trompetenabschätzung). Die Existenz der Zustandstrajektorie ist auf dem ganzen Intervall gesichert.
\item Die Integration auf den Teilintervallen ist entkoppelt und kann parallel erfolgen, in Realtime somit schneller als Single-Shooting.
\item Die Blockbandstruktur der Hessematrix erlaubt Updates von höherem Rang und damit bessere Approximation der Hessematrix der Lagrange-Funktion und schnellere Konvergenz
\item das QP hat zwar mehr Variablen und mehr Nebenbedingungen, aber auch Strukturen. Werden diese z.\,B. durch Kondensierung ausgenutzt ist der Aufwand zum Lösen des QP im wesentlichen auch nicht größer als beim Single-Shooting.
\eitm

Bemerkung: Auch Kollokation kann für die direkte Methode der Optimalsteuerung eingesetzt werden, analog zu Kapitel 4(2).





















