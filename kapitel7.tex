\chaptr{SQP-Verfahren}

Zunächst: gleichungsbeschränkter Fall

\[ \min f(x) \text{ s.\,t. } g(x) = 0 \quad (7.1) \]

Die notwendigen Optimalitätsbedinungen erster Ordnung lauten:

\begin{align*}
\nabla_x L(x,\lambda) &= \nabla f(x) + \nabla g(x) \lambda = 0 \quad (7.2) \\
\lambda_\lambda L(x,\lambda) &= g(x) = 0 \\
\text{bzw. } \nabla L(x,\lambda) = 0 \\
\end{align*}

Das ist ein nichtlineares Gleichungssystem in $(x,\lambda)$. Dieses wollen wir mit dem Newton-Verfahren lösen. Ein Schritt $(\Delta x, \Delta \lambda)$ erfüllt dabei das LGS

\[ \nabla L(x,\lambda) + \nabla^2 L(x,\lambda) \bpm \Delta x \\ \Delta \lambda \epm = 0 \]

(Iterationsindex $k$ weggelassen) bzw.

\[ \bpm \nabla f(x) + \nabla g(x) \lambda \\ g(x) \epm      +     \bpm \nabla_{xx}^2 L(x,\lambda) & \nabla g(x) \\ \nabla g(x)^T & 0 \epm      \bpm \Delta x \\ \Delta \lambda \epm     =    \bpm 0 \\ 0 \epm \]

\msection{Lemma 7.1}

Sei $\Delta x$ die Lösung des QP

\[ \min \frac 12 \Delta x^T H(x) \Delta x + \nabla f(x)^T \Delta x \text{ s.\,t. } 0 = A(x) \Delta x + g(x) \quad (7.5) \]

mit Matrixfunktionen $A$, $H$.

Dann existiert ein $\Delta \lambda$, so dass für beliebige $\lambda$

\[ \bpm \nabla f(x) + A(x)^T \lambda \\ g(x) \epm   +   \bpm H(x) & A(x)^T \\ A(x) & 0 \epm   \bpm \Delta x \\ \Delta \lambda \epm   = 0 \quad (7.6) \]

Beweis: Lagrangefunktion des QP mit Multiplikator $u$

\begin{align*}
L(\Delta x, u) &= \frac 12 \Delta x^T H \Delta x + \nabla f^T \Delta x + u^T(A\Delta x + g) \\
\nabla_{\Delta x} L(\Delta x, u) &= H\Delta x + \nabla f + A^T u - A^T \lambda + A^T \lambda &= 0 \\
\nabla_u L(\Delta x, u) &= A \Delta x + g \\
\bpm \nabla f + A^T \lambda \\ g \epm   +   \bpm H & A^T \\ A & 0 \epm   \bpm \Delta x \\ u-\lambda \epm   &= 0 \\
\end{align*}

mit $\Delta \lambda = u-\lambda$.

Umgekehrt:

\msection{Lemma 7.2}

Falls $\Delta \lambda$ existiert, so dass $\bpm \Delta x & \Delta \Lambda \epm^T$ die Gleichung (7.6) erfüllt und wenn $H(x)$ positiv definit auf $\ker A(x)$, dann ist $\Delta x$ Minimum von (7.5).

\msection{Algorithmus 7.3: SQP-Verfahren für gleichungsbeschränkte Probleme}

\bitm
\item Startwert $x^0$, ggf. $\lambda^0$, $j:=0$
\item Solange ein Abbruchkriterium verletzt ist:
\bitm
\item Berechne $\Delta x^j$ als Lösung des folgenden QPs und ggf. die Lagrangemultiplikatoren $u^j$:
\[ \min_{\Delta x} \frac 12 \Delta x^T H^j \Delta x + \nabla f(x^j)^T \Delta x \text{ s.\,t. }0 = A^j \Delta x + g(x^j) \]
mit
\[H^j \cong \nabla_{xx}^2 L(x^j, \lambda^j), A^j \cong \nabla g(x^j)^T \]
\item Iteriere:
\begin{align*}
x^{j+1} :&= x^+ + \alpha^j \Delta x^j \\
\lambda^{j+1} :&= \lambda^k + \alpha^j (\underbrace{u^j-\lambda^j}_{\Delta \lambda^j}) \\
\end{align*}
mit $\alpha^j \in (0,1]$ aus einer Globalisierungsstrategie (z.\,B. Linesearch).
\eitm
\eitm

Bemerkung: $\lambda^j$ wird benötigt zur Berechnung von $H^j$ (siehe unten) und zur Ungleichungsbehandlung. Wenn $\alpha^j=1$ hängt $\lambda^{j+1}$ nicht von $\lambda^j$ ab.

\msection{Korollar 7.4}

Für die Wahl $H^j = \nabla_{xx}^2 L(x^j, \lambda^j)$ und $A^j = \nabla g(x^j)^T$ ist das SQP-Verfahren ein Newton-Verfahren für die KKT-Bedingung des gleichungsbeschränkten NLP. Nach Korollar 5.4 konvergiert es lokal quadratisch, wenn $\alpha^j \equiv 1 \forall j \geq \ov j$.

\msection{Bemerkung 7.5}

Für $H^j \cong \nabla_{xx}^2 L(x^j, \lambda^j)$ und $A^j = \nabla g(x^j)^T$ ist das SQP-Verfahren ein Quasi-Newton-Verfahren, genannt Partial-Quasi-Newton-SQP-Verfahren. Wenn außerdem $A^j \cong \nabla g(x^j)^T$ nennt man das Verfahren Total-Quasi-Newton-SQP-Verfahren.

Die Konvergenzrate hängt von der Wahl der Approximationen ab.

\msection{Lösung von QPs mit Gleichungsbeschränkungen}

\[ \min_p \frac 12 p^T H p + g^T p \text{ s.\,t. } Ap+b= 0 \quad (7.7) \]

mit $p\in\R^n$, $A\in \R^{m\times n}$, $m\leq n$, $\Rg a = m$, $H \in \R^{n\times n}$ symmetrisch und positiv definit auf $\ker A$, $g\in \R^n$, $b\in \R^m$.

\bitm
\item zulässige Menge ist konvex
\item Zielfunktion ist konvex auf der zulässigen Menge
\eitm

$\RA$ es existiert genau ein Minimum.

Äquivalent:

\[ \exists u: \bpm H & A^T \\ A & 0 \epm    \bpm p \\ u \epm    =    - \bpm g \\ b \epm \]

Variante 1: Bildraummethode: Numerisch instabil, im Allgemeinen teurer als Variante 2, nur geeignet wenn $H$ insgesamt positiv definit.

\bitm
\item Multipliziere die erste Blockzeile mit $H^{-1}$ und $A$ und ziehe von der zweiten Blockzeile ab:
\[ \bpm H & A^T \\ 0 & - A H^{-1} A^T \epm   \bpm p \\ u \epm   =   -\bpm g \\ b - AH^{-1} g \epm \]
erfordert eine Cholesky-Zerlegung von $H$ und eine Matrixmultiplikation.
\item Zerlege $-AH^{-1}A^T$ (sog. Schur-Komplement), berechne $u$
\item Aus erster Blockzeile:
\[ p = H^{-1}( -g + A^T u) \]
\eitm

Variante 2: Nullraummethode: stabil, deutlich schneller als Variante 1, wenn $m > \frac n2$.

\bitm
\item QR-Zerlegung von $A^T$: 
\begin{align*}
A^T &= Q^T \tilde L^T = \bpm \underbrace{Q_1^T}_{n\times m} & \underbrace{Q_2^T}_{n\times (m-n)} \epm   \bpm L^T \\ 0 \epm = Q_1^T L^T  \\
y :&= Qp = \bpm Q_1 p \\ Q_2 p \epm = \bpm y_1 \\ y_2 \epm \\
Ap &= \bpm L & 0 \epm   \bpm Q_1 \\ Q_2 \epm p = L Q_1 p \\
&= L y_1 = -b \\
y_1 &= - L^{-1} b
\end{align*}
$y_2$ zunächst frei.
\[AQ_2^T = LQ_1 Q_2^T = 0 \]
Spalten von $Q_2^T$ bilden eine Basis von $\ker A$.
\[ p = Q^T y = Q_1 y_2 + Q_2^T y_2 \]
\item Einsetzen in die erste Blockzeile und Multiplikation mit $Q$:
\begin{align*}
QHQ^T y + QA^T u &= -Qg \\
\end{align*}
hat zwei Teile:
\bitm
\item $A_2 H (Q_1^T y_1 + Q_2^T y_2) + \underbrace{Q_2 A^T}_{=0} u = -Q_2 g $ bzw. $\underbrace{Q_2 H Q_2^T}_{(*)} y_2 = -Q_2(y-H Q_1^T L^{-1} b)$ (*): auf $\ker A$ projeziertes $H$, positiv definit, zerlege mit Cholesky $\RA$ $y_2$ und $p=Q^T y$
\item $Q_1 (Hp + g) + Q_1 A^T u = 0$, $Q_1 A^T = L^T$ $\RA$ $u = -L^{-T} Q_1 (Hp + g)$.
\eitm
\eitm

\msection{7.3: Quasi-Newton-SQP mit Update}

Ziel: berechne $H^k \cong \nabla_{xx}^2 L(x^k, \lambda^k)$ "`gut und billig"'.

Niedrigrang-Aufdatierungen:

\[H^{k+1} = H^k + \overbrace{\underbrace{ab^T}_{\text{hat Rang } 1} + cd^T}^{\text{hat Rang 2}} \]

mit $a,b,c,d\in \R^n \setminus\{0\}$

\msection{Wichtigstes Beispiel: BFGS (Broyden, Fletcher, Goldfarb, Shanno)}

\[ H^{k+1} = H^k + \frac{y^k {y^k}^T}{{y^k}^T y^k} - \frac{(H^k s^k)(H^k s^k)^T}{{s^k}^T H^k s^k} (7.9) \]

mit

\begin{align*}
y^k &= \nabla_x L(x^{k+1}, \lambda^{k+1}) - \nabla_x L(x^k, \lambda^{k+1}) \\
s^k &= x^{k+1} - x^k = \alpha^k \Delta x^k \\
\end{align*}

\msection{Definition 7.5: Sekantenbedingung}

$H^{k+1}$ erfüllt die Sekantenbedingung, wenn

\begin{align*}
H^{k+1} s^k &= y^k \quad (7.10)
\end{align*}

Das bedeutet, dass $H^{k+1}$ in erster Ordnung korrekt ist. Taylorentwicklung um $x^{k+1}$ angewendet bei $x^k$

\begin{align*}
\nabla_{xx}^2 L(x^{k+1}, \lambda^{k+1}) (x^{k+1} - x^k) &= \nabla_x L(x^{k+1}, \lambda^{k+1}) - \nabla_x L(x^k, \lambda^{k+1}) + \mathcal O(\|x^{k+1} - x^k\|^2) \\
\nabla_{xx}^2 L(x^{k+1}, \lambda^{k+1}) s^k &= y^k + \mathcal O(\|x^{k+1}-x^k\|^2) \\
\end{align*}


\msection{DFP (Davidon, Fletcher, Powell)}

\begin{align*}
H^{k+1} &= H^k + U_{DFP} (H^k, y^k, s^k) \\
w^k :&= y^k - H^k s^k \\
&= H^k + \frac{w^k (y^k)^T + y^k (w^k)^T}{(y^k)^T s^k} - \frac{(s^k)^T w^k}{(y^k)^T s^k} \frac{y^k (y^k)^T}{(y^k)^T s^k} \quad (7.11) \\
\end{align*}

\msection{Eigenschaften von BFGS und DFP}

\bitm
\item BFGS und DFP-Updates erfüllen die Sekantenbedingung.
\item Ist $H^k$ symmetrisch, dann ist auch die BFGS- bzw. DFP-Approximation $H^{k+1}$ symmetrisch.
\item Ist $(y^k)^T s^k > 0$ (7.12) und $H^k$ positiv definit, dann ist auch die BFGS- bzw. DFP-Approximation $H^{k+1}$ positiv definit. Wenn $(y^k)^T s^k \geq 0$ kann man das Update dämpfen: $\tilde H^{k+1} = (1-\alpha) H^k + \alpha H^{k+1}$ mit $\alpha < 1$ geeignet. $\tilde H^{k+1}$ erfüllt dann nicht mehr die Sekantenbedingung!.
\eitm

\msection{Satz 7.7}

Die Inverse des BFGS-Updates ist der DFP-Updates mit $y^k$ und $s^k$ vertauscht:

\begin{align*}
H^{k+1} &= H^k + U_{BFGS} (H^k, y^k, s^k) \text{ und } \\
B^{k+1} &= B^k + U_{DFP}  (B^k, s^k, y^k) \text{ Dann gilt: } \\
\text{Ist } H^k B^k &= I \text{ dann ist auch } \\
H^{k+1} B^{k+1} &= I \\
\end{align*}

In der Praxis kann man die Updates auch direkt auf die Zerlegungen anwenden.

\msection{Resultate zur lokalen Konvergenz von BFGS-Quasi-Newton-SQP-Verfahren}

\msection{Satz 7.8: Dennis-Mor\'e im beschränkten Fall}

Sei $x^*$ ein lokales Minimum, $x^*$ regulär, gelte die hinreichende Bedingung zweiter Ordnung. Seien $f,g \in C^2$ mit Lipschitz-stetigen zweiten Ableitungen. Im SQP-Verfahren werde die Hessematrix $\nabla_{xx}^2 L$ durch eine Quasi-Approximation $H^k$ angenähert. Die Folge $\{x_k\}$ konvergiere gegen $x^*$. Die Konvergenz ist genau dann superlinear, wenn

\[ \lim_{k \to \infty} \frac{\|P^k (H^k - \nabla_{xx}^2 L(x^*)) (x^{k+1} - x^k)\|}{\|x^{k+1} - x^k\|} = 0 \]

Dabei ist $P^k$ eine Matrix, die auf dem Nullraum von $\nabla {g^k}^T$ projeziert.

\msection{Bemerkung 7.9}

\[ P^k = I - \nabla g^k (\nabla {g^k}^T \nabla g^k)^{-1} \nabla {g^k}^T \]

bzw. $P^k = Z^k {Z^k}^T$, wenn die Spalten von $Z^k$ den Nullraum von ${\nabla g^k}^T$ aufspannen, z.\,B. $\nabla g^k = \bpm Q_1^T & Q_2^T \epm \bpm L \\ = \epm$, $Z^k = Q_2^T$.

Nebenrechnung:

\begin{align*}
p &= p_1 + p_2 \\
p_1 &= \nabla g q \\
p_2 &= Q_2^T r \\
p^k p &= Q_2^T Q_2  \\
&= Q_2^T \underbrace{Q_2 \nabla g}_{=0} q + \underbrace{Q_2^T Q}_{=I} Q_2^T r \\
&= Q_2^T r = p_2 \\
\end{align*}


\msection{Satz 7.10 (Superlineare Konvergenz von BFGS-SQP-Verfahren}

Gelten die Annahmen wie in Satz 7.8. Seien $\nabla_{xx} L(x^*)$ und $H^0$ symmetrisch und positiv definit. Wenn $\|x^0 - x^*\|$ genügend klein und $\|\nabla_{xx}^2 L(x^*) - H^0\|$ genügend kleine sind erfüllen die BFGS-Approximationen $H^k$ die Formel (7.12). Die Iterierten konvergieren superlinear gegen $x^*$. Dieses Resultat gilt auch für Ungleichungsbeschränkungen.

\msection{7.4 SQP-Verfahren für NLPs mit Ungleichungsbeschränkungen}

\[ \min f(x) \text{ s.\,t. } g(x) = 0 \text{ und } h(x) \leq 0 \quad (7.13) \]

SQP-Verfahren: wie Algorithmus 7.3 aber löse ungleichungsbeschränkte QPs:

\begin{align*}
\min &\frac 12 \Delta x^T H^k \Delta x + \nabla f(x^k)^T \Delta x  \quad (7.14) \\
0 &= g(x^k) + \nabla g(x^k)^T \Delta x \\
0 & \geq h(x^k) + \nabla h(x^k)^T \nabla x \\
\end{align*}

mit Lösung $\Delta x^k$ und Lagrange-Multiplikatoren $\mu^k$ und $v^k \geq 0$. Iteriere $x^{k+1}$, $\lambda^{k+1}$ und $\mu^{k+1} := \mu^k + \alpha^k (v^k - \mu^k)$. Wenn $\mu^0 \geq 0$ und $\alpha \in (0,1]$ ist $\mu^k \geq 0$.


\msection{Lemma 7.11}

Sei $x^*$ Lösung von 7.13. Dann gilt:

\begin{align*}
\min &\frac 12 \Delta x^T H \Delta x + \nabla f(x^*)^T \Delta x \quad (7.15)\\
0 &= g(x^*) + \nabla g(x^*)^T \Delta x \\
0 & \geq h(x^*) + \nabla h(x^*)^T \Delta x
\end{align*}

mit $H$ positiv definit auf $T(x^*) = \{ p: \nabla g(x^*)^T pi = 0, \nabla h_i (x^*)^T p = 0, i \in I(x^*) \}$ hat die Lösung $\Delta x = 0$ und die Lagrangemultiplikatoren $u = \lambda^*$, $v = \mu^*$. Die aktiven Ungleichungen des QP sind $I(x^*)$. Beweis:

\begin{align*}
L_{QP} &= \frac 12 \Delta x^T H \Delta x + \nabla {f^*}^T \Delta x + u^T (g^* + \nabla {g^*}^T \Delta x) + v^T (h^* + \nabla {h^*}^T \Delta x) \\
\nabla_{\Delta x} L_{QP} &= H \Delta x + \nabla f^* + \nabla g^* u + \nabla h^* v = 0 \\
\end{align*}

für $\Delta x = 0$, $u = \lambda^*$, $v = \mu^*$ wegen Minimalität von $x^*$. Gleichungs- und Ungleichungsbeschränkungen sind erfüllt, $v \geq 0$.

\msection{Lemma 7.12}

Sei $\hat x \in U_\eps (x^*)$ für geeignetes $\eps>0$. Sei $H$ positiv definit auf $T(x^*)$. Betrachte das QP (7.15) mit $\hat x$ statt $x^*$. Für dessen Lösung gilt $\hat x \cong 0$, $\hat \mu \cong \lambda^*$, $\hat v \cong \mu^*$ und die aktiven Ungleichungen sind $I(x^*)$. Beweis: Störungssatz 6.27.

\msection{Folgerung 7.13}

\bitm
\item Für festes $H$ werden die aktiven Indizes bestimmt, wenn $\hat x \to x^*$.
\item Wenn das SQP-Verfahren konvergiert: $x^k \to x^*$ und die Folge der $H^k$ beschränkt bleibt, dann existiert $k_0$, so dass $\forall k \geq k_0$ das QP (7.14) die aktiven Indizes $I(x^*)$ liefert.
\eitm
% Wenn man so ein SQP-Verfahren beim Arbeiten beobachtet...

\msection{Lösung von Ungleichungsbeschränkten QP mit einer Active-Set-Methode}

QP: Schreibe $x$ statt $\Delta x$, eigentlich Unterproblem des SQP-Verfahrens. Annahme: $H$ positiv definit.

\begin{align*}
\min &\frac 12 x^T H x + g^T x \text{ (quadratische Zielfunktion)} \\
\text{ s.\,t. } b_i &= a_i^T x, i \in E \text{ (Gleichungsbeschränkungen)} \\
b_i &\geq a_i^T x, i \in I \text{ (Ungleichungsbeschränkungen)} \\
\end{align*}

$A(x) := \{ i \in E \cup I : a_i^T x = b \}$ Indexmenge alle mit Gleichheit erfüllten Beschränkungen. Wenn man $A(x^*)$ kennt kann man die lösung $x^*$ durch lösen des Gleichungsbeschränkten QPs

\[ \min \frac 12 x^T H x + g^T x \text{ s.\,t. } b_i = a_i^T x, i \in A(x^*) \]

bestimmen.

\msection{Active-Set-Strategie}

hier: primale Variante.

Starte mit einer Schätzung $W^0$ des active Set, ändere $W^k$ ("`Working Set"') sukzessive. $W^k$ besteht aus allen Gleichungsbeschränkungen und einigen Ungleichungsbeschränkungen, so dass immer die $a_i^T$, $i \in W^k$ linear unabhängig sind.

\msection{Algorithmus 7.14: Active-Set-Methode für konvexe QP}

\bitm
\item Starte mit einem zulässigen Startpunkt $x^0$. Sei $W^0$ eine Teilmenge von $A(x^*)$ mit linear unabhängigen $a_i^T$ und sei $k:=0$.

Iteriere $k=0,1,2,\cdots$
\item Bestimme Schritt $p^k$ durch Lösung von  $\min \frac 12 p^T H p + {c^k}^T p$ so dass $0 = a_i^T p$, $i \in W^k$ mit $c^k = H x^k + g$. $p^k$ erfüllt $a_i^T (x^k + \alpha p^k) = a_i^T x^k = b_i \forall i \in W^k$, d.\,h. Schritte in Richtung $p^k$ bleiben zulässig $\forall i \in W^k \forall \alpha \in \R$.
\item Wen $p^k = 0$: Berechne Lagrangemultiplikatoren $\lambda^k$ für (7.17). Diese erfüllen
\begin{align*}
0 &= H p^k + x^k + \sum\limits_{i\in W^k} \lambda_i^k a_i \\
\RA - \sum\limits_{i\in W^k} \lambda_i^k a_i &= c^k = H x^k + g \\
\end{align*}
\item  Wenn $\lambda_i^k \geq 0 \forall i\in W^k \cap I$ und $\lambda_i^k := 0$ für $i \notin W^k$, dann sind alle Optimalitätsbedingungen des QP (7.16) erfüllt:
\begin{align*}
H x^k + g + \sum\limits_{i \in E \cup I} \lambda_i^k a_i &= 0 \\
a_i^Tx^k &= b_i \quad i \in E \\
\lambda_i^k & \geq 0 \quad \forall i \in I \\
a_i^T x^k &= b_i \quad i \in I \\
\end{align*}
$x^* := x^k$ ist Lösung, Stop.
\item Wenn $\lambda_i^k < 0$ für ein $j \in W^k$, dann kann man durch Weglassen dieser Nebenbedingung den Wert der Zielfunktion verkleinern. Ändere den Working set $W^{k+1} := W^k \setminus \{j_0 \}$ mit $j_0\colon$ ein Index des kleinsten $\lambda_i^k < 0$. Dann gilt: Im nächsten Schritt bleibt die Nebenbedingung $j_0$ zulässig: $x^{k+1} = x^k$, $p^{k+1}$, $\lambda^{k+1}$ erfüllt
\begin{align*}
-\sum\limits_{i\in W^{k+1}} \lambda_i^{k+1} a_i &= H p^{k+1} + c^{k+1} \\
&= H p^{k+1} + H x^k + g \\
\text{und } a_i^T p^{k+1} &= 0 \quad \forall i \in W^{k+1} \\
\text{Ergibt: } -\sum\limits_{i\in W^{k1+}} (\lambda_i^{k+1} - \lambda_i^k) a_i + \lambda_{j_0}^k a_{j_0} &= H p^{k+1} \\
\lambda_{j_0}^k {p^{k+1}}^T a_{j_0} &= {p^{k+1}}^T H p^{k+1} \geq 0 \\
\text{und somit } a_{j_0}^T (x^{k+1} + \alpha p^{k+1}) &= a_{j_0}^T x^k + \alpha a_{j_0}^T p^{k+1} \leq b_{j_0} \quad \forall \alpha \geq 0 \\
\end{align*}
\item Wenn $p^k \neq 0$: Bestimme Schrittweite $\alpha^k \in [0,1]$, so dass für $x^{k+1} = x^k + \alpha^k p^k$ alle Nebenbedingungen, auch $i \notin W^k$ erfüllt sind.
Falls $a_i^T p^k \leq$ für $i \notin W^k$ dann ist $a_i^T (x^k + \alpha p^k) = a_i^T x^k + \alpha a_i^T p^k \leq b_i \, \forall \alpha \geq 0$.
Wenn $a_i^T p^k > 0$ muss $\alpha \leq \frac{b_i - a_i^T x^k}{a_i^T p^k}$ erfüllt sein. Wähle $\alpha^k := \min \l\{1,\underbrace{\min_{i\notin W^k,\,a_i^T p^k > 0} \frac{b_i-a_i^T x^k}{a_i^T p^k}}_{(*)} \r\}$ (*): Nebenbedingungen, die das erfüllen heißen blockierende Nebenbedingungen. $\alpha^k$ kann auch $0$ sein. Wenn es blockierende Nebenbedingungen gibt, bilde $W^{k+1} := W^k \cup \{i_0\}$ mit einer blockierenden NB $i_0$. Ansonsten setze $W^{k+1} = W^k$. Iteriere $X^{k+1} := x^k + \alpha^k p^k$. Dabe gilt: Die Zielfunktion nimmt in Richtung $p^k$ strikt ab: Weil $p^k$ minimal ist für (7.17) gilt:
\[ \frac 12 {p^k}^T H p^k + {c^k}^T p^k < \frac 12 0^T H 0 + {c^k}^T 0 = 0 \]
also
\begin{align*}
\frac 12 (x^k + \alpha p^k)^T H (x^k + \alpha p^k) + g^T (x^k + \alpha p^k) \\
&= \frac 12 {x^k}^T H x^k + g^T x^k + \alpha \underbrace{ {x^k}^T p^k}_{<0} + \frac 12 \alpha^2 \underbrace{{p^k}^T H p^k}_{>0} \\
& < \frac 12 {x^k}^T H x^k + g^T x^k
\end{align*}
für $\alpha$ klein genug. ggf. verkleinere $\alpha$ durch Linesearch
\eitm

\msection{Bemerkung 7.15}

In jeder Iteration bleiben alle Nebenbedingungen erfüllt: Die, die in $W^k$ bleiben, siehe Schritt 1. Die, die nicht in $W^k$ waren siehe Schritt 5. Die, die aus $W^k$ entfernt wurden siehe Schritt 4.

\msection{Satz 7.16}

Unter bestimmten Vorraussetzungen terminiert Algorithmus 7.14 in endlich vielen Iterationen.

Alternative:

\msection{Innere-Punkt-Methode}

Phase 1: Löse ein Hilfs-QP (LP) zur Bestimmung eines zulässigen $x^0$.



















