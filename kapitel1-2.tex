\section*{Adjungierte Differentialgleichung}

Will man Matrix-Matrix-Produkte von links an $\tfrac{\partial y}{\partial y_0}$ oder $\tfrac{\partial y}{\partial p}$ berechnen:

\begin{align*}
u^T \frac{\partial y}{\partial y_0} & & \text{ mit } u\in\R^{n_y} \text{ oder} \\
U^T \frac{\partial y}{\partial y_0} & & \text{ mit } U \in \R^{n_y \times n}
\end{align*}

Berechnet man ebenfalls nicht zuerst $\frac{\partial y}{\partial y_0}$ und multipliziert dann, sondern löst die adjungierte Differentialgleichung.

\subsection*{Satz 1.12}

Gegeben sei das ODE-AWP

\[ \dot y = f(t,y,p), \quad y(t_0) = y_0 \quad (1.28) \]

Dann gilt: Integriert man die adjungierte Differentialgleichung (ADE)

\[ \dot \Lambda (t)^T = -\Lambda(t)^T \frac{\partial f}{\partial y} (t,y,p) \quad (1.29) \]

rückwärts, d.\,h. ausgehend von $T > t_0$ mit dem Anfangswert $\Lambda(T) = I$, dann gilt:

\begin{align*}
\frac{\partial y}{\partial y_0} (T) &= \Lambda (t_0)^T & (1.30) \\
\frac{\partial y}{\partial p} (T) &= \int\limits_{t_0}^T \Lambda(t)^T \frac{\partial f}{\partial p} (t,y,p) \dd t & (1.31)
\end{align*}

\emph{Beweis:}

Es gilt:

\begin{align*}
\dot y - f(t,y,p) &= 0 \\
\RA \int\limits_{t_0}^T \Lambda T (\dot y - f(t,y,p)) \dd t &= 0 & (1.32)
\end{align*}

Leite (1.32) nach $y_0$ ab:

\begin{align*}
0 &= \frac{\partial}{\partial y_0} \int\limits_{t_0}^T \Lambda^T (\dot y -f(t,y,p)) \dd t \\
&= \int\limits_{t_0}^T \Lambda^T \l( \frac{\partial \dot y}{\partial y_0} - \frac{\partial f}{\partial y} \frac{\partial y}{\partial y_0} \r) \dd t
\end{align*}

Partielle Integration:

\[ \int\limits_{t_0}^T \Lambda^T \frac{\partial \dot y}{\partial y_0} \dd t = \l[ \Lambda^T \frac{\partial y}{\partial y_0} \r]_{t_0}^T - \int\limits_{t_0}^T \dot \Lambda^T \frac{\partial y}{\partial y_0} \dd t \]

Daraus folgt:

\begin{align*}
0 &= \int\limits_{t_0}^T \Lambda^T \l( \frac{\partial \dot y}{\partial y_0} - \frac{\partial f}{\partial y_0} \frac{\partial y}{\partial y_0} \r) \dd t \\
&= \int\limits_{t_0}^T \l( -\dot \Lambda - \Lambda^T \frac{\partial f}{\partial y} \frac{\partial y}{\partial y_0} \r) \dd t + \l[\Lambda^T \frac{\partial y}{\partial y_0} \r]_{t_0}^T \\
&= \int\limits_{t_0}^T \underbrace{\l( -\dot \Lambda^T - \Lambda^T \frac{\partial f}{\partial y} \r)}_{= 0 \text{ adj. DGL}} \frac{\partial y}{\partial y_0} \dd t + \underbrace{\Lambda(T)^T}_{=I \text{ (AW)}} \frac{\partial y}{\partial y_0}(T) - \Lambda(t_0)^T \underbrace{\frac{\partial y}{\partial y_0}}_{=I} \\
&= \frac{\partial y}{\partial y_0} (T) - \Lambda(t_0)^T \\
\RA \Lambda^T(t_0) &= \frac{\partial y}{\partial y_0} (T) 
\end{align*}

Leite (1.32) nach $p$ ab mit analoger partieller Integration:

\begin{align*}
0 &= \int\limits_{t_0}^T \Lambda^T \l( \frac{\partial \dot y}{\partial p} - \frac{\partial f}{\partial y} \frac{\partial y}{\partial p} - \frac{\partial f}{\partial p} \r) \dd t \\
&= \int\limits_{t_0}^T \underbrace{\l( -\dot \Lambda^T - \Lambda^T \frac{\partial f}{\partial y} \r)}_{= 0} \frac{\partial y}{\partial p} \dd t - \int\limits_{t_0}^T \Lambda^T \frac{\partial f}{\partial p} \dd t + \underbrace{\Lambda^T}_{=I} \frac{\partial y}{\partial p}(T) - \Lambda^T(t_0) \underbrace{\frac{\partial y}{\partial p}(t_0)}_{=0} \\
&= -\int\limits_{t_0}^T \Lambda^T \frac{\partial f}{\partial p} \dd t + \frac{\partial y}{\partial p} (T) \\
\RA \frac{\partial y}{\partial p} (T) &= \int\limits_{t_0}^T \Lambda^T \frac{\partial f}{\partial p} \dd t
\end{align*}

\subsection*{Bemerkung 1.13}

Zur Berechnung von $\Lambda^T$ muss man erst vorwärts (1.28) lösen und $y$ berechnen, die Werte von $y$ dabei zwischenspeichern und dann rückwärts (1.29) lösen, um $\Lambda^T$ zu berechnen.

\subsection*{Ableitung von Linearkombinationen von $\frac{\partial y}{\partial y_0}$}

Gegeben: $u\in\R^{n_y}$ (adjungierte Richtung). Die Linearkombinations-ADE

\[ \partial_t (u^T \Lambda(t)^T) = - (u^T \Lambda (t)^T) \frac{\partial f}{\partial y} (t,y,p) \quad (1.23) \]

mit dem Anfangswert

\[ (u^T \Lambda(T)^T) = u^T \quad (1.34) \]

hat die Lösung

\begin{align*}
u^T \Lambda(t_0)^T &= u^T \frac{\partial y}{\partial y_0} (T) & (1.35) \\
\text{und } \int\limits_{t_o}^T (u^T \Lambda(t)^T) \frac{\partial f}{\partial p} (t,y,p) \dd t &= u^T \frac{\partial y}{\partial p} (T) & (1.36)
\end{align*}

D.\,h. pro Linearkombination der Zeilen von $\tfrac{\partial y}{\partial y_0}$ und $\tfrac{\partial y}{\partial p}$ muss nur eine Rückwärts-AWP gelöst werden.

\subsection*{Anwendung:}

\begin{align*}
& \Phi(y(T; t_0, y_0, p)) & & \text{ "`Zielfunktion"'} \\
& \Phi: \R^{n_y}  \to \R
\end{align*}

Gradient:

\[ \nabla_p \Phi(y(T; t_0, y_0,p)) = \frac{\partial \Phi}{\partial y} \frac{\partial y}{\partial p} (T; t_0,y_0,p) \]

benötigt nur eine adjungierte Richtung $\tfrac{\partial \Phi}{\partial y} \in \R^{n_y}$

\subsection*{Zusammenfassung:}

\begin{itemize}
\item Variationsdifferentialgleichung
\begin{itemize}
	\item Vorwärtsdifferentiation
	\item Richtungsableitungen
\end{itemize}
\item Adjungierte Differentialgleichung
\begin{itemize}
	\item Rückwärtsdifferentiation
	\item Linearkombinationen
\end{itemize}
\end{itemize}

\section*{Simulation und Optimierungsprobleme bei Differentialgleichungen}

% Bild siehe Seite 21 in Kapitel1.pdf

\begin{itemize}
\item \emph{Simulation:} Löse die Mathematischen Modellgleichungen, in dieser Vorlesung: Integration von ODE/DAE-AWPn
\item \emph{Optimierungsprobleme:}
\begin{itemize}
	\item \emph{Parameterschätzung:} Bestimme die Modellparameter $p$ so, dass Modell und Realität möglichst gut übereinstimmen.
	\item \emph{Modelldiskriminierung:} Bestimme durch Experimente, welche Modellvariante die Realität besser beschreibt.
	\item \emph{Optimale Versuchsplanung:} Bestimme Experimente, aus denen die Parameter mögliuchst signifikant geschätzt werden können.
	\item \emph{Optimales Design:} Berechne, wie ein System, Gerät etc. nach einem bestimmten Ziel optimal gebaut werden soll.
	\item \emph{Optimale Steuerung:} Berechne, wie ein Prozess nach einem bestimmten Ziel optimal durchgeführt werden soll, typischerweise: Minimale Kosten oder maximale Ausbeute.
	\item \emph{Optimale modellbasierte Regelung:} Wie ändert sich die optimale Steuerung bei Störungen des Systems?
\end{itemize}
\end{itemize}


