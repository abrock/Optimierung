
\chaptr{8. Optimale Steuerung}

\msection{Problemformulierung}

\msubsection{Beispiel: Fahrdynamik}

Bringe Fahrzeug (Bahn, Auto) von A nach B. Fahrzeug soll am Anfang und Ende in Ruhe sein. Minimiere dabei die Zeit oder den Energieverbrauch oder maximiere die Reichweite etc.

Modell: Gewöhnliche Differentialgleichung:

\begin{align*}
\ddot s &= \frac{\partial^2}{\partial t^2} s = \frac 1M f(s,\dot s, u) \\
s \colon & [0,T] \to \R^n \quad \text{ "`Ort"'} \\
v = \dot s \colon &[0,T] \to \R^n \quad \text{ "`Geschwindigkeit"'} \\
\ddot s \colon &[0,T] \to \R^n \quad \text{ "`Beschleunigung"'} \\
M: & \text{ Masse} \\
f: & \text{ Kräfte, hängen ab von }\\
- & \text{ Ort $s$, z.\,B. Gefälle} \\
- & \text{ Gewschwindigkeit (z.\,B. Reibung)} \\ 
- & \text{ Steuerung $u$, z.\,B. Gaspedal, Bremse, Lenkrad} \\
\end{align*}

Steuerungen sind in Grenzen frei wählbar.

Spezialfall: Raketenwagen

\begin{align*}
\ddot s &= u \\
\underline u & \leq u(t) \leq \ov u \quad \text{ Steuerungsbeschränkung} \\
\end{align*}

Randbedingungen:

\begin{align*}
s(0) &= A \\
s(T) &= B \\
\dot s(0) &= v(0) = 0 \\
\dot s(T) &= v(T) = 0 \\
\end{align*}

Transformation auf System erster Ordnung:

\begin{align*}
\dot v &= u \\
\dot s &= v
\end{align*}

Zeitintervall $[0,T]$. Die Endzeit $T$ kann fest oder frei sein.

Zustandsbeschränkung: $v_{min} \leq v(t) \leq v_{max}$. Erste Frage: Gibt es zulässige Lösungen, also Steuerungen $u$ und Zustände $(s,v)$ so dass die Differentialgleichungen, Randbedingungen, Randbedingungen, Steuerungsbeschränkungen und Zustandsbeschränkungen erfüllt sind? Sinnvolles Setting im Beispiel: $\underline u < 0 < \ov u$. $T$ muss ausreichend groß sein. $v_{min} \leq 0$, zumindest lokal im Ort, $v_{max} > 0$. Wenn $T$ genügend groß ist gibt es im Allgemeinen mehrere Lösungen.

Wenn $T$ so klein wie möglich gewählt wird, gibt es in der Regel nur eine zulässige Lösung.

Zweite Frage: Es gebe zulässige Lösungen, suche optimale Lösung.

Mögliche Kriterien:

\bitm
\item zeitoptimal: $\min T = \int_0^T 1 \dt$
\item energieoptimal: $\min \frac 12 \int_0^T u(t)^2 \dd t$ 
\item reichweitenoptimal: $\max s(T)$, lasse Randbedingungen $s(t)=B$, $v(T) = 0$ weg, zusätzliche Nebenbedingung: $\frac 12 \int_0^T u(t)^2 \dd t \leq E_{max}$.
\item Kombinationen dieser Kriterien: $\min w_1 T + w_2 \frac 12 \int_0^T u(t)^2 \dd t$.
\eitm

\msection{Allgemeine Problemformulierung}

\msubsection{Modellgleichung}

\begin{align*}
\dot y(t) &= f(t,y(t),q,u(t)) \quad (8.1) \\
y\colon [t_0, \tend] &\to \R^{n_y}: \quad \text{Zustände} \\
q & \in \R^{n_q}: \quad \text{Steuergrößen}
u\colon [t_0, \tend] &\to \R^{n_u} \quad \text{Steuerfunktionen, Steuerungen} \\ 
\end{align*}

Bemerkung: Steuerungen und Steuergrößen können vom Experimentator gewählt werden. $t_0$ und $\tend$ können fest oder frei sein.

\msubsection{Randbedingungen}

\begin{align*}
y(t_0) &= y_0 \quad \text{Anfangsbedingung} \\
r(t_0, y(t_0), t_1, y(t_1), \cdots, \tend, y(\tend), q) &= 0 \quad \text{ oder } \geq 0 \\
\end{align*}


\msubsection{Steuerungsbeschränkungen}

\begin{align*}
u(t) & \in \Omega(t) \subset \R^{n_u} \\
\Omega(t): & \quad \text{Steuerungsbereich} \\
\end{align*}

$u$ ist messbar.

Beispiele:

\bitm
\item keine Beschränkungen
\item $\Omega$ Polyeder, $\Omega = \{ u : a_j^T u \leq b_j, j=1,\cdots,m \}$
\item $\Omega$ Quader, $\Omega = \prod_{i=1}^{n_u} [\underline u_i, \ov u_i ]$
\item $\Omega$ diskret, $\Omega = \{ 1,2,3,4,R\}$ (Gangschaltung o.\,ä.).
\item Kombinationen davon
\eitm


\msubsection{Zustandsbeschränkungen}

\begin{align*}
b(t,y(t), q, u(t)) &= 0 \quad \text{ oder } \geq 0 \quad \forall t \in [t_0, \tend] 
\end{align*}

\msubsection{Zielfunktional}

\bitm
\item Mayersche Form: \[ \min_{u,q,y,\tend} \Phi(y(\tend), \tend, q) \]
\item Lagrange Form: \[ \min_{u,q,y,\tend} \int\limits_{t_0}^\tend L(t,y(t),q,u(t)) \dd t \]
\item Bolza-Form: \[ \min_{u,q,y,\tend} \Phi(y(\tend), \tend, q) + \int\limits_{t_0}^\tend L(t,y(t),q,u(t)) \dd t \]
\eitm

\msubsection{Bemerkung 8.1}

Statt einer ODE kann auch eine DAE betrachtet werden. Dann sind die Konsistenzbedingungen zu berücksichtigen, siehe oben.

\msubsection{Bemerkung 8.2}

Da die Steuerfunktion im Allgemeinen beliebige messbare Funktionen sein können, ergeben Punktauswertungen von $u$ keinen Sinn. $u(t)$ geht deshalb nicht in die Randbedingung und nicht in den Meyer-ZF-Term ein.

\msubsection{Bemerkung 8.3}

Die drei Standardformen des Zielfunktionals sind äquivalent und können ineinander übergeführt werden, siehe Übungsaufgabe.

\msubsection{Bemerkung 8.4}

Probleme mit freier Endzeit können in Probleme mit fester Endzeit transformiert werden:

Definiere normierte Zeit 

\begin{align*}
\tau &\in [0,1] \\
t &= t_0 + \tau (\tend-t_0) = \phi(\tau) \\
\dd t &= (\tend -t_0) \dd \tau =: \hat t \dd \tau \\
\text{also } \frac{\dd \phi}{\dd \tau} &= \tend - t_0 = \hat t \\
\text{Dann ist } y(t) &= y(\phi(\tau)) =: \ov y(\tau)  \\
u(t) &= u(\phi(\tau)) =: \ov u(\tau) \\
\frac{\dd \ov y}{\dd \tau} &= \frac{\dd y}{\dd t} \cdot \frac{\dd \phi}{\dd \tau} = f(\phi(\tau), y(\phi(\tau)), q, u(\phi(\tau))) \cdot \hat t \\
&=: \ov f(\tau, \ov y(\tau), \ov q, \ov u(\tau)) \\
\text{mit } \ov q &= (q, \hat t) \\
\end{align*}

Die Intervalllänge $\hat t$ ist zu einer zusätzlichen Steuergröße geworden. Das so transformierte DGL-System hat das feste Zeitintervall $\tau \in [0,1]$. $\hat x$, $\hat u$ setzt man in Zielfunktion, Randbedingungen, Steuerungebeschränkungen und Zustandsbeschränkungen ein.

\msection{8.3 Ansätze zur Lösung von Optimalsteuerungsproblemen}

\msubsection{First Optimize then Discretize}

Formuliere die Optimalitätsbedinungen des Optimalsteuerungsproblems im Funktionenraum. Hier: Pontryaginsches Maximumsprinzip. Erhalte Randwertproblem. "`Rate"' Struktur der optimalen Lösung. Verifiziere Optimalität. Vorteile:

\bitm
\item Rigorose mathematische Untersuchung der optimalen Lösung.
\eitm

Nachteile:

\bitm
\item Für komplexe Probleme schwierig zu formulieren.
\item Struktur der Lösung mus erraten werden.
\eitm

Siehe Kapitel 10: Indirekter Ansatz.


\msubsection{First Discretize then Optimize}

Parametrisiere die Steuerfunktion in einem endlichdimensionalen Unterraum. Erhalte endlichdimensionales, ggf. hochdimensionales aber strukturiertes Optimierungsproblem. Löse diese mit maßgeschneiderten NLP-Methoden.

Nachteile:

\bitm
\item Es werden nur suboptimale Lösungen berechnet.
\item Strukturelle Aussagen sind schwer möglich.
\eitm

Vorteile:

\bitm
\item In der Praxis werden Steuerungen oft genau so realisiert, wie sie hier parametrisiert werden.
\item Die Methode kann sehr flexibel und robust 
\item Zur Simulation der Modellgleichungen, Auswertung der Ableitungen und zur nichtlinearen Optimierung können maßgeschneiderte numerische State-of-the-art-Methoden eingesetzt werden, siehe Kapitel 9: Direkter Ansatz.
\eitm

